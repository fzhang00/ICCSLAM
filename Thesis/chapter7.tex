\chapter{Conclusion}\label{ch:conclusion}
% word checked.
This thesis described a research that addressed obstacle detection
problem for low flying UAV using monocular camera. The research
focused on static obstacle detection for medium size UAV deployed in
natural environment and performing low altitude terrain following
flight. An EKF based SLAM algorithm was proposed in chapter
\ref{ch:algorithm}. The algorithm utilized multiple sensors and
inverse depth parametrization to detect long distanced objects that
range up to a couple thousands meters. A flight test was performed to
collect aerial video and sensor recordings to test the performance of
the algorithm. In addition, to evaluate the accuracy of the algorithm,
digital terrain map of the survey zone was processed to compare to the
result from the algorithm. Further analysis of the algorithm and error
source was performed through a series of simulation which simulated the
motion seen in flight and camera calibration error. Both flight and
simulation results were summarized in this chapter. Suggestions on
future work were also described at the end. 

\section{Result Summaries}
\subsection{Flight Data Processing}
% word checked.
The flight data was processed, and results were discussed in chapter
\ref{ch:FlightResult}. Sparse corner features were extracted from the
flight image sequence, and were tracked by the CC\_EKF\_SLAM algorithm
to estimates landmarks positions. An estimated terrain map and the
trajectory of the SUAS were generated from the video sequence. The
estimated terrain map was compared to the downloaded DEM, and the SUAS
trajectory was compared to the on-board GPS and magnetometer
recording. Results from the flight data are as followed:
\begin{itemize}
  \item For SUAS localization accuracy, position error accumulated
  gradually, and reached a maximum of 20 meters on X axis, 50 meters
  on Y axis and 30 meter on Z axis at the end of the video sequence.
  The error for orientation estimates were within $1.15^\circ$ for all
  three components.
  \item For landmarks mapping,  most landmarks tracked by the
  algorithm had their inverse depths converged to stable values, with the
  furthest landmark located at 1600 meters from SUAS location at 1$^{st}$.
  \item The accuracy of the estimated landmarks locations achieved an
  average error of -56 meters on X axis, 13 meters on Y axis and 1.3
  meters on Z axis.
  \item Landmarks added after 1$^{st}$ frame showed offset errors on Y
  axis. This was due to the error in the SUAS localization and the
  propagation of this error into the landmarks mapping through
  reference frame transformation from camera frame to world frame.
\end{itemize}

Due to the difficulty in corresponding the tracked landmarks with DEM,
an extra video was processed with the airport buildings in the scenes
with which manual visual correspondence was possible. All landmarks
were matched manually to the satellite images from Google Earth. In
this test, landmarks positions had errors of 40 meters on Y and 20
meters on Z components. Given that they were all located at the corner
of the image plane, this error was likely to be due to lens
distortion, which produced up to 400 meters of error on X, up to 150
meters of error for Y, and up to 100 of meters error for Z according
to simulation.

\subsection{Noise Analysis through Simulation}
% word checked.
To better understand noise source in CC\_EKF\_SLAM algorithm, a series
of simulations were done with variable error settings for the camera
motions, camera intrinsic parameters, and image resolutions. A nearly
no noise environment was first simulated with UAV moving forward only.
No error was introduced from image digitization or camera model
mismatch. Effects from various UAV motions were simulated next. Then
various camera model mismatch and image resolution settings were
simulated. The discussion of the simulation results was presented in
chapter \ref{ch:simulation}.

\subsubsection{Low noise environment}
% word checked.
Under forward only motion, UAV position errors were less than 1cm, and
orientation errors was under $3e-3^\circ$. Landmarks positions errors
were under 0.2 meters on X axis, and under 0.02 meters on Y and
Z axes. The result showed error introduced by the algorithm
itself under simple forward motion was very minor.

\subsubsection{Impact from motion}
% word checked.
The effects from complex UAV motions were analyzed next. The UAV
remained forward traveling while the other 5 D.O.F. motion was added
one at a time. The added motion was simulated with a 1Hz sine wave and
with variable amplitude setting. Results showed that translational
motion on Y axis and Z axis and rotation around X axis (roll) has
little effect on the UAV localization accuracy. Rotation around Y axis
(pitch) and rotation around Z axis (azimuth) caused UAV positioning
errors up to 30 meters peak-to-peak on Z and Y respectfully depending
on the amplitude of the rotation. For landmarks  mapping accuracy,
translation motion only increased landmarks positions errors by
centimeters; rotational motion increased landmarks positions errors by
meters for rotation around X, and hundreds of meters for rotation
around Y and Z. Most errors appeared on landmarks added after first frame,
caused by an initialization offset on parameter $\varphi^W$ originated
from the error in UAV localization.

\subsubsection{Impact from error in camera calibration} 
Second part of chapter \ref{ch:simulation} analyzed impact from the
error in camera calibration. Results were summarized below:

\paragraph{Effect from ($c_x, c_y$)}
% word checked.
\begin{itemize}
  \item Calibration error in $c_x$ or $c_y$ caused UAV position estimates to diverge from the correct value with time.
  \item Calibration error in $c_x$ or $c_y$ caused error in landmarks
  positions estimates on all three axes. $c_x$ affected the X and Y
  component, and $c_y$ affected X and Z component. The amount of error
  was related to the distances of the landmarks from camera, and the
  amount of error in $c_x$ or $c_y$ through first order polynomial equation.
\end{itemize}

\paragraph{Effect from ($f_x, f_y$)}
% word checked.
\begin{itemize}
  \item Calibration error in $f_x$ or $f_y$ did not affect UAV
  localization estimates.
  \item Calibration error in $f_x$ affected the landmarks positions on
  Y component, error in $f_y$ affected the Z component. The amount of
  error in landmarks positions estimates were dependable on the actual
  landmarks distances from camera, and the amount of error in $f_x$
  or $f_y$ through first order polynomial equation.
\end{itemize}

\paragraph{Effect from Distortion}
% word checked.
\begin{itemize}
  \item Ignoring distortion brought errors into all six parameters of
  UAV localization. The position error on X axis was diverging
  from the true value with time. All other parameters oscillated around
  zero, but the amount of error grew bigger in time.
  \item Errors on landmarks mapping resulted from ignoring the
  distortion were related to the landmarks positions on Y and Z axes.
  In another word, the further the landmark positioned from the
  optical center, the greater the error became.
\end{itemize}

\paragraph{Effect from Image Resolution}
% word checked.
\begin{itemize}
  \item To achieve good accuracy on both self localization and landmarks
  mapping, higher image resolution is a must.
  \item The 480x720 resolution used in the flight test resulted in
  landmarks mapping error of +/-150 meters on X axis.
  \item Increasing resolution to 720x1080 significantly reduced the
  variance on both localization and mapping estimates.
\end{itemize}

\section{Future Research}
% word checked.
This research presented a SLAM framework for obstacle detection
through a monocular camera. Flight test proved the feasibility of the
algorithm for mapping landmarks as far as more than 1 thousand meters.
To make the algorithm practical in real survey environment, a number
of improvements can be made to increase the its accuracy and reliability.
\begin{enumerate}
  \item In the current implementation, lens distortion was ignored. As
  the simulation suggested, this introduced significant error to
  landmarks positioned close to the edge of FOV. Including lens
  distortion into the measurement model should improve the accuracy of
  the algorithm.
  \item Sensor resolution should be increased should future flight
  test opportunity arises.
  \item A filter for checking corner feature quality of the landmark
  should be added. Landmarks that have poor corner signature lead to
  greater tracking error. In addition, due to the accumulated tracking
  error, the corner feature of landmark can be completely different
  from what it was first extracted after a few seconds of tracking.
  Since there exist correlations between landmarks, these tracking
  error has impact on other landmarks and UAV localization estimates.
  Removing the bad landmarks, and replacing them with good quality one is
  necessary to make the algorithm more robust.
  \item To enable large area mapping, a map joining algorithm should
  be added.
  \item Localization error has a big impact on the accuracy of the
  map. To minimize localization error, when GPS data is available,
  which is the case most of the time, all parameters should be
  synchronized to GPS periodically. This procedure should also help to
  prevent consistency problem. GPS syncing could be integrated with
  the map joining algorithm.
\end{enumerate}

Besides improvements listed above, the cause of sensitivity of the
algorithm to rotational motion on Y and Z axes should be studied
further, and a method of compensating for the error should be
developed.

%%% Local Variables:
%%% mode: latex
%%% TeX-master: "thesis"
%%% End:
