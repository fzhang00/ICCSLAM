\chapter{Conclusion}\label{ch:conclusion}

This thesis described the research that address obstacle
detection problem for low flying SUAS using monocular camera. The
research focused on static obstacle detection for medium size SUAS
deployed in natural environment and performing low altitude terrain
following flight.  An EKF based SLAM algorithm was
proposed in chapter \ref{ch:algorithm}. The algorithm utilized
multiple sensors and inverse depth parametrization to detect distanced
object that range up to a couple thousands meters away. A flight test
was performed to collect aerial video and sensor reading for testing
the performance of the algorithm. In addition, to evaluate the
accuracy of the algorithm, digital terrain map of the survey zone was
processed to compare to the result from the algorithm. 

The flight data was processed, and result was discussed in chapter
\ref{ch:FlightResult}. Sparse corner features were extracted from the
flight image sequence, and were tracked by the CC\_EKF\_SLAM
algorithm. An estimated terrain map and the trajectory of the SUAS is
generated from the video sequence. The estimated terrain map was
compared to the downloaded DEM, and the SUAS trajectory was compared
to the on-board GPS recording. Result from the flight data are as
followed:
\begin{itemize}
  \item Analysis of the result shows that most features tracked by the
  filter has their inverse depth converged to a stable value, with the
  furtherest feature at 1600 meters.
  \item For the accuracy of the estimated feature location, X
  components achieve a maximum error of 120 meters; for Y and Z
  components, error converged to less than 20 meters for features
  initialized at first frame.
  \item Feature added during tracking shows a offset error for its Y
  component estimation. This is due to the error in the SUAS
  localization error and the propagation of this error into the
  features mapping through the camera frame to world frame
  transformation.
  \item For SUAS localization accuracy, coordinate error on X
  component has a maximum of 20 meters, 50 meters on Y component and
  30 meter on Z component. The error for orientation estimates are
  within $1.15^\circ$ for all three components. .
\end{itemize}

Due to the difficulty of corresponding the extracted feature to the
DEM, an extra video was processed with the airport buildings in the
scenes. All features was corresponded to the satellite images with eye
that's available through Google Earth. In this test, features position
has 100 meters error on both X and Y components. Given that they are
located at the corner of the image plane, this error was likely to be
the result of lens distortion, which shows up to 400 meters error X
component, up to 150 meters error for Y component, and up to 100
meters error for Z component. 

To better understand noise source in CC\_EKF\_SLAM algorithm, a series
of simulation was done with variable error setting for the camera
intrinsic parameters. The detail of the simulation was presented in
chapter \ref{ch:simulation}. 

An nearly no noise environment was first simulated with SUAS moving
forward only, with no error introduced from image digitization or
camera model mismatch. Effect from various SUAS motion was simulated
next. Then various image resolution settings, and camera model
mismatch were simulated. Under forward only motion, SUAS translation
error is less than 1cm, orientation error under $3e-3^\circ$; feature
position error is under 0.2 meters for X component, and under 0.02
meters for Y and Z components. The result above shows error introduced
by the algorithm itself under simple forward motion is very minor.

The effect from complex SUAS motion was analyzed next. The SUAS remain
forward traveling. The other 5 d.o.f. motion was simulated with a 1Hz
sine wave with variable amplitude setting, and added on top of the
forward motion one at a time. Result shows that translational motion
and rotation around X axis has little effect on the SUAS localization
accuracy. Rotation around Y axis (pitch) and rotation around Z axis
(azimuth) causes SUAS positioning error on Z and Y respectfully to up to
30 meters peak-to-peak depending on the magnitude of the rotation. For
feature mapping accuracy, translation motion

%%% Local Variables:
%%% mode: latex
%%% TeX-master: "thesis"
%%% End:
