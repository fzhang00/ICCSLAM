\chapter{Conclusion}\label{ch:conclusion}
% word checked.
This thesis describes a research that addresses the obstacle detection
problem for a low flying UAV using a monocular camera. The research
focused on static obstacle detection for a medium size UAV deployed in
a natural environment and performing a low altitude terrain following
flight. An EKF based SLAM algorithm CC-EKF-SLAM was proposed in
Chapter \ref{ch:algorithm}. The algorithm utilizes multiple sensors
and inverse depth parametrization to detect long distance objects
that range up to a thousand meters. A flight test was performed to
collect aerial video and sensor recordings, accomplished by towing a
sensor-loaded SUAS with a helicopter. Two pieces of aerial video and
data were processed by the CC-EKF-SLAM algorithm. Convergence and
consistency characteristics of the algorithm were studied. In addition, to
evaluate the accuracy of the results, a digital terrain map of the
survey zone was processed to compare to the results from the algorithm.
Further analysis of the algorithm and error sources was performed
through a series of simulations which simulated UAV motion as seen in
flight and camera calibration errors. Both flight and simulation
results are summarized in this chapter. Suggestions on future work
are also described at the end.

\section{Result Summaries}
\subsection{Test Results from Real Flight Data}
% word checked.
The flight data was processed, and results were discussed in Chapter
\ref{ch:FlightResult}. Sparse corner features were extracted from the
flight image sequence, and were tracked by the CC-EKF-SLAM algorithm
to estimate landmark positions. An estimated terrain map and the
trajectory of the SUAS were generated from the video sequence. The
estimated terrain map was compared to the downloaded DEM, and the SUAS
trajectory was compared to the on-board GPS and magnetometer
recording. Results from the flight data are as follows:
\begin{itemize}
  \item For SUAS localization accuracy, position error accumulates
  gradually, and reaches a maximum of 20 meters on the X axis, 50 meters
  on the Y axis and 30 meters on the Z axis at the end of the video sequence.
  The error for orientation estimates are within $1.15^\circ$ for all
  three components.
  \item For landmark mapping, most landmarks tracked by the
  algorithm have their inverse depths converged to stable values, with the
  furthest landmark located at 1600 meters from SUAS location at the
  first frame.
  \item The accuracy of the estimated landmark locations achieve an
  average error of -56 meters on the X axis, 13 meters on the Y axis
  and 1.3 meters on the Z axis.
  \item Landmarks added after the first frame show offset errors on
  the Y axis. This is due to the error in the SUAS localization and
  the propagation of this error into the landmark mapping through
  reference frame transformation from the camera frame to the world
  frame.
\end{itemize}

Due to the difficulty in matching the tracked landmarks with the DEM
in natural scene, an extra video was processed with the airport
buildings in the scenes which makes manual visual correspondence
possible. All landmarks were matched manually to the satellite image
from Google Earth. In this test, landmark positions had errors of 150
meters on the X axis, 40 meters on the Y axis and 20 meters on the Z
axis. Given that they are all located at the corner of the image
plane, this error is likely due to lens distortion.

\subsection{Noise Analysis through Simulation}
% word checked.
To better understand noise source in the CC-EKF-SLAM algorithm, a series
of simulations were done with variable settings for the camera
motions, camera intrinsic parameters, and image resolutions. A nearly
no noise environment was first simulated with the UAV moving forward only.
Effects from various UAV motions were simulated next. Then
various camera model mismatch and image resolution settings were
simulated. The discussion of the simulation results was presented in
Chapter \ref{ch:simulation}.

\subsubsection{Low noise environment}
% word checked.
Under forward only motion, UAV position errors are less than 1cm, and
orientation errors are under $3e-3^\circ$. Landmark position errors
are under 0.2 meters on the X axis, and under 0.02 meters on the Y and
Z axes. The result shows error introduced by the algorithm
itself under simple forward motion was very minor.

\subsubsection{Impact from motion}
% word checked.
The effects of oscillatory UAV motions were analyzed next. The UAV
remained forward motion while oscillatory motion on each of the other
5 DOFs was added one at a time. The added motion was simulated with a
1Hz Sine wave and with variable amplitude settings. Results show that
translational motion on the Y axis and the Z axis and rotation around X axis
(roll) have little effect on the UAV localization accuracy. Rotation
around the Y axis (pitch) and rotation around the Z axis (heading) cause UAV
positioning errors up to 30 meters peak-to-peak on Z and Y
respectively, depending on the amplitude of the rotation. For landmark
mapping accuracy, translation motion only increase landmark
position errors by centimeters; rotational motion increase landmark
position errors by meters for rotation around X, and hundreds of
meters for rotation around Y and Z. Most errors appear on landmarks
added after the first frame, caused by an initialization offset on
parameter $\varphi^W$ originating from the errors in UAV localization.

\subsubsection{Impact from error in camera calibration} 
The second part of Chapter \ref{ch:simulation} analyzed the impact from the
error in camera calibration. Results are summarized below:

\paragraph{Effect of Error in ($c_x, c_y$)}
% word checked.
\begin{itemize}
  \item Calibration error in $c_x$ or $c_y$ causes UAV position estimates to diverge from the correct value with time.
  \item Calibration error in $c_x$ or $c_y$ causes errors in landmark
  position estimates on all three axes. $c_x$ affects the X and Y
  components, and $c_y$ affects the X and Z components. The amount of
  error is related to the locations of the landmarks, and the amount
  of error in $c_x$ or $c_y$ through the first order polynomial equations.
\end{itemize}

\paragraph{Effect of Error in ($f_x, f_y$)}
% word checked.
\begin{itemize}
  \item Calibration error in $f_x$ or $f_y$ has little effect on UAV
  localization estimates.
  \item Calibration error in $f_x$ affects the landmark positions on the
  Y component, error in $f_y$ affects the Z component. The amount of
  error in landmark position estimates are dependent on the actual
  landmark locations and the amount of error in $f_x$
  or $f_y$ through the first order polynomial equations.
\end{itemize}

\paragraph{Effect of Distortion}
% word checked.
\begin{itemize}
  \item Ignoring distortion brings errors into all six parameters of
  UAV localization. The position error on the X axis is diverging
  from the true value with time. All other parameters oscillate around
  zero, but the amount of error grows bigger with time.
  \item Errors in landmark mapping resulted from ignoring the
  distortion are related to the landmark positions on the Y and Z
  axes. The further the landmark is positioned from the optical
  center, the greater the error becomes.
\end{itemize}

\paragraph{Effect of Image Resolution}
% word checked.
\begin{itemize}
  \item To achieve good accuracy on both self localization and landmark
  mapping, higher image resolution is a must.
  \item The 480x720 resolution used in the flight test could result in
  landmark mapping errors of up to +/-150 meters on the X axis.
  \item Increasing resolution to 1080x1440 significantly reduces the
  variance of errors on both localization and mapping estimates.
\end{itemize}

\section{Future Research}
% word checked.
This research has presented a SLAM framework for obstacle detection
through a monocular camera. Flight test proved the feasibility of the
algorithm for mapping landmarks more than 1 thousand meters away.
To make the algorithm practical in a realistic survey environment, a number
of improvements can be made to increase its accuracy and reliability.
\begin{enumerate}
  \item Including lens distortion into the measurement model should
  improve the accuracy of the algorithm, as suggested by the error
  analysis.
  \item Sensor resolution should be increased if a future flight
  test opportunity arises.
  \item A filter for checking the quality of corner features should be
  added. Landmarks that have poor corner signatures lead to greater
  tracking errors. In addition, due to the accumulated tracking error,
  the visual pattern of a landmark can be completely different from
  what it was when first extracted. Since there are correlations between
  landmarks, these tracking errors have an impact on other landmarks and
  UAV localization estimates. Removing the bad landmarks, and
  replacing them with good quality ones is necessary to make the
  algorithm more robust.
  \item To enable large area mapping, a map joining algorithm should
  be added.
  \item Localization error has a big impact on the accuracy of the
  map. To minimize localization error, when GPS data is available,
  which is the case most of the time, all parameters should be
  synchronized to GPS periodically. This procedure should also help to
  prevent consistency problems. GPS syncing could be integrated with
  the map joining algorithm.
\end{enumerate}

Besides the improvements listed above, the cause of the sensitivity of
the algorithm to rotation on the Y and Z axes should be
studied further, and a method of compensating for the error should be
developed.

%%% Local Variables:
%%% mode: latex
%%% TeX-master: "thesis"
%%% End:
