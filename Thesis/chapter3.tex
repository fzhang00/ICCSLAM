\chapter{Algorithm Description}

The algorithm described in Chapter!!!Unresolved reference!!!3 was 
implemented in Python programming language. An open source machine 
vision library OpenCV was utilized to perform feature extraction and 
tracking. The feature extraction method used was the Shi-Tomasi corner 
detector. Feature tracking was accomplished through the pyramid 
implementation of Lucas-Kanade optical flow method . 

\section{Camera Centric Inverse Depth Parameterization}
The standard way of describing a feature's position is to use the 
Euclidean XYZ parameterization. In practical outdoor range estimation 
problem, the algorithm must deal with features located at near infinity. 
Inverse depth parameterization overcomes this problem by representing 
range $d$ in its inverse form $\rho =1/d$. In addition, features at 
infinity contribute in estimating the camera rotational motion even 
though they offer little information on camera translational motion. 
Furthermore, inverse depth parameterization allows us to initialize 
features into the EKF framework before it is safely triangulated.

The inverse depth parameterization used in this work was first 
introduced in . All features and camera positions are referred to a 
world reference frame. When used in an Extended Kalman Filter framework, 
the system suffers decreasing linearity when camera is moving away from 
the world origin. A modified method which uses the camera center as the 
origin was proposed in . Our work has adopted the camera centered 
approach with minor modification to integrate the inertial measurements.

% \begin{figure}[h]
% \centering
% \includegraphics[width=250.55pt,height=109.45pt]{media/image2.eps}
% \caption{Figure 2 Inverse Parameterization}
% \label{fig:algo1}
% \end{figure}

A scene point $p_{i}^{C}$ can be defined by 6 parameters, with the 
superscript $C$ representing a camera reference frame. ():
\begin{equation}
p_{i}^{C}=\begin{bmatrix}
x_{i}^{C} & y_{i}^{C} & z_{i}^{C} & \rho _{i} & \varphi _{i}^{C} & 
\theta _{i}^{C} 
\end{bmatrix}
\end{equation}
The first three parameters $[x_{i}^{C}, y_{i}^{C}, z_{i}^{C}]$ 
represent the initial position where the feature is first observed. $\rho _{i}$ is the inverse distance from the initialization position to 
the feature. The elevation-azimuth pair $[\phi_{i}^{C}, \theta_{i}^{C}]$ encodes a unit vector pointing from the initialization point 
to the feature. The vector is given by
\begin{equation}
m(\phi_{i}^{C}, \theta_{i}^{C})=\begin{bmatrix}
\cos\varphi_{i}^{C}\cos\theta _{i}^{C} \\
\cos\varphi_{i}^{C}\sin\theta _{i}^{C} \\
\sin\varphi_{i}^{C}
\end{bmatrix}
\end{equation}

\section{Modeling the System with Extended Kalman 
Filter}

\subsection{Full State Vector}

The EKF state vector is defined as 

\begin{equation}
x=\begin{bmatrix}
OX_{W}^{C} & c^{C} & r^{C} & p_{1}^{C} & p_{2}^{C} & \ldots 
\end{bmatrix}
\end{equation}

\noindent where $OX_{W}^{C}= \begin{bmatrix}O_{x}^{C} & O_{y}^{C} &
  O_{z}^{C} & W_{x}^{C} & W_{y}^{C} & W_{z}^{C} \end{bmatrix}^{T}$
contains translation parameters $O_{x,y,z}^{C}$ and rotation
parameters $W_{x,y,z}^{C}$ to transform the camera reference frame to
the world reference frame. $\left(c^{C},r^{C}\right)^{T}$ represents
the camera translation and rotation motion frame by frame in Euclidean
coordinates, and $p_{i}^{C}$ contains the feature parameters as
described in the previous section.

\subsection{Prediction} (Will be updated to a more complete form)

For a prediction step at time $k$, the world frame and features
parameters are kept unchanged from time $k-1$. The camera parameters
are updated using the new inertial measurements: velocity $v^{C}$,
acceleration $a^{C}$, and rate of change in roll/pitch/yaw $w^{C}$.
The camera motion parameters at time $k$ are then
\begin{multline}
\begin{bmatrix}
OX_{W}^{C} & c^{C} & r^{C} & p_{1}^{C}& p_{2}^{C} & \vdots & p_n^C
\end{bmatrix}_{k}^T \\
=\begin{bmatrix}
OX_{W,k-1}^{C} & c_{measured}^{C} & r_{measured}^{C} & p_{1,k-1}^{C} &
p_{2,k-1}^{C} & \vdots & p_{n,k-1}^C
\end{bmatrix}_{k-1}^T
\end{multline}

Where 

$$c_{measured}^{C}=v_{measured}^{C}\Delta t+ \frac{1}{2}a_{measured}^{C}\Delta t^{2}$$
$$r_{measured}^{C}=r_{k-1}^{C}+ w_{measured}^{C}$$

\subsection{Measurement Model}

Each observed feature is related to the camera motion through the 
measurement model (). This relationship enables a correction on the 
camera motion and features parameters based on the features' locations 
observed in the image. 

For a feature $p_{i}^{C}$, the vector $h^{R}$pointing from the 
predicted camera location to the feature initialization position is 

\begin{equation}
h_{k}^{R}=\begin{bmatrix}
x_{i}^{C} \\
y_{i}^{C} \\
z_{i}^{C} \\
\end{bmatrix}_{k}-\begin{bmatrix}
c_{x}^{C} \\
c_{y}^{C} \\
c_{z}^{C} \\
\end{bmatrix}_{k}
\end{equation}

The normalized vector pointing from the predicted camera position to the 
feature at time k is then 

\begin{equation}
  h_{k}^{C}=Q^{-1}\left(r_{k}^{C}\right)\left(\rho _{k}h_{k}^{R}+m\left(\varphi_{ 
        k}^{C},\theta _{k}^{C}\right)\right)
\end{equation}

\noindent where $Q^{-1}(r_{k}^{C})$ is the inverse rotation matrix from the 
camera frame at time $k-1$ to camera frame at time $k$. From vector 
$h_{k}^{C}$, the feature location on image plane can be found by

\begin{equation}
h_{k}^{U}= \begin{bmatrix}
u_{k} \\
v_{k} \\
\end{bmatrix}=\begin{bmatrix}
\frac{s_{x}h_{y,k}^{C}}{h_{x,k}^{C}} \\
\frac{s_{y}h_{z,k}^{C}}{h_{x,k}^{C}} \\
\end{bmatrix}
\end{equation}

where $s_{x}$ and $s_{y}$ is the scaling factor of the projection, 
obtained through camera calibration.

% \begin{figure}[h]
% \centering
% \includegraphics[width=252.3pt,height=157.25pt]{media/image3.eps}
% \caption{Measurement Model}
% \label{ch3fig:3}

\subsection{Composition Step}

Update step corrects the camera motion and feature location in camera 
frame k-1. To continue to the next cycle of tracking, all parameter must 
be transform to camera frame k. World reference point coordinate and 
orientation from k-1 to k is related by

\begin{equation}
\begin{bmatrix}
O_{x}^{C_{k}} \\
O_{y}^{C} \\
O_{z}^{C} \\
\end{bmatrix}_{k}=R^{-1}(r_{k}^{C_{k-1}})\left(
\begin{bmatrix}
O_{x}^{C_{k-1}} \\
O_{y}^{C_{k-1}} \\
O_{z}^{C_{k-1}} \\
\end{bmatrix}_{k}- \begin{bmatrix}
c_{x}^{C_{k-1}} \\
c_{y}^{C_{k-1}} \\
c_{z}^{C_{k-1}} \\
\end{bmatrix}_{k}\right)
\end{equation}

\begin{equation}
\begin{bmatrix}
W_{x}^{C_{k}} \\
W_{y}^{C_{k}} \\
W_{z}^{C_{k}} \\
\end{bmatrix}_{k-1}= \begin{bmatrix}
W_{x}^{C_{k-1}} \\
W_{y}^{C_{k-1}} \\
W_{z}^{C_{k-1}} \\
\end{bmatrix}_{k}-r^{C_{k-1}}
\end{equation}

Feature parameters in new camera frame are related to the previous frame 
by

\begin{equation}
\begin{bmatrix}
x_{i}^{C_{k}} \\
y_{i}^{C_{k}} \\
z_{i}^{C_{k}} \\
\end{bmatrix}_{k}=Q^{-1}(r^{C_{k-1}})\left(
\begin{bmatrix}
x_{i}^{C_{k-1}} \\
y_{i}^{C_{k-1}} \\
z_{i}^{C_{k-1}} \\
\end{bmatrix}_{k}- \begin{bmatrix}
c_{x}^{C_{k-1}} \\
c_{y}^{C_{k-1}} \\
c_{z}^{C_{k-1}} \\
\end{bmatrix}_{k}\right)
\end{equation}

\begin{equation}
\begin{bmatrix}
\rho_{i} \\
\varphi_{i}^{C_{k}} \\
\theta_{i}^{C_{k}} \\
\end{bmatrix}_{k}=
\begin{bmatrix}
\rho _{i, k} \\
m^{-1}(R^{-1}(r^{C_{k-1}})m(\varphi _{i, k}^{C_{k-1}}, \theta _{i, k}^{C_{k-1}}) \\
\end{bmatrix}
\end{equation}

\noindent where $m(\varphi_{i, k}^{C_{k-1}}, \theta_{i, k}^{C_{k-1}})$ is the 
unit vector pointing from the initialization point to the feature seen 
by the camera at step $k-1$

The covariance matrix is also affected by this transform. Therefore must 
be updated. The new covariance matrix is related to the old one by 

\begin{equation}
P_{k}^{C_{k}}=J_{C_{k-1}\to C_{k}}P_{k}^{C_{k-1}}J_{C_{k-1}\to C_{k}}^{T}
\end{equation}

The calculation of $J_{C_{k-1}\to C_{k}}$ is the same as the 
linearization of prediction matrix in section Method 2.

In order to apply the correction and update the camera reference frame
to the new camera position, an additional composition step is
necessary. The world reference frame parameters and features
parameters are updated by applying reference frame transformation from
the camera location at time $k-1$ to camera location at time k. The
EKF covariance matrix $P_{k}$ is also updated through

\begin{equation}
P_{k}=JP_{k}J^{T}
\end{equation}

\noindent where $J$ is the Jacobian of the composition equations. 
\section{Initialization}
\subsection{Initialize the State Vector}

State vectors are initialized at the first frame. The world origin 
coordinate and orientation, camera motions, and the feature reference 
points are all initialized to zeros, with variance equals to the 
smallest machine number. Thus, 

\begin{equation}
OX_{W}^{C}=\begin{bmatrix}0&0&0&0&0&0\end{bmatrix} 
\end{equation}

\begin{equation}
c^{C}=\begin{bmatrix}0&0&0\end{bmatrix}
\end{equation}

\begin{equation}
r^{C}=\begin{bmatrix}0&0&0\end{bmatrix}
\end{equation}

\begin{equation}
p_{i}^{C}=\begin{bmatrix}0&0&0&\rho _{i}&\varphi_{i}&\theta_{i}\end{bmatrix}
\end{equation}

The inverse distance $\rho $ of all features are initialized to 0.1 
because we are dealing with long distance object. The features 
elevation-azimuth pair $[\varphi _{i}^{C}, \theta _{i}^{C}]$ is 
extracted from features coordinates in image plane. First, a vector 
pointing from camera optical center to feature can be defined by

\begin{equation}
h^{C}=\begin{bmatrix}
h_{x}^{C}\\
h_{y}^{C}\\
h_{z}^{C}\\
\end{bmatrix}
 = \begin{bmatrix}
1 \\
u\cdot s_{x} \\
v\cdot s_{y} \\
\end{bmatrix}
\end{equation}

Where $[u v]$ is the feature coordinate in the image, and $
[s_{x}s_{y}] $is the scaling factor of the projection from the scene to 
image plane. The elevation-azimuth pair $[\varphi _{i}^{C}, \theta 
_{i}^{C}]$ can then be directly calculated from $h^{C}$

\begin{equation}
\varphi 
=arctan\left(\frac{h_{z}^{C}}{\sqrt{{h_{x}^{C}}^{2}+{h_{y}^{C}}^{2}}}\right)
\end{equation}

\begin{equation}
\theta =arctan\left(\frac{h_{y}^{C}}{h_{x}^{C}}\right)
\end{equation}


\subsection{Initialized the State Covariance Matrix}

Because the world origin is defined at the first frame, it enables 
initializing the filter with minimum variance, which helps reducing the 
lower bound of the filter error. The covariance matrix of the world 
coordinate and orientation, and the camera motion is 

\begin{equation}
P=I_{12\times 12}\cdot \epsilon 
\end{equation}

where $I$ is a $12\times12$ identity matrix, and $\epsilon $ is the 
lowest significant bit (LSB) of a machine.

The covariance of features is added one by one as there is 
correlation between them. For every new feature added, the new 
covariance matrix becomes

\begin{equation}
P_{new}=J\begin{bmatrix}
P_{old} & 0 \\
0 & R \\
\end{bmatrix}
J^{T}
\end{equation}

\noindent where $P_{old}$ isthe covariance matrix of the existing state vector, 
and the initial $P_{old}$ is defined in . Matrix R is the covariance 
matrix of the variable in features initialization.

\begin{equation}
R=\begin{bmatrix}
\sigma _{x_{i}^{C}} & & & & & & \\
 & \sigma _{y_{i}^{C}} & & & 0 & & \\
 & & \sigma _{z_{i}^{C}} & & & & \\
 & & & \sigma _{\rho } & & & \\
 & 0 & & & \sigma _{image} & & \\
 & & & & & \sigma _{image} & \\
\end{bmatrix}
 = \begin{bmatrix}
\epsilon & & & & & & \\
 & \epsilon & & & 0 & & \\
 & & \epsilon & & & & \\
 & & & 0.1 & & & \\
 & 0 & & & 1 & & \\
 & & & & & 1 & \\
\end{bmatrix} 
\end{equation}

\noindent where$[\sigma _{x_{i}^{C}}\sigma _{y_{i}^{C}}\sigma
_{z_{i}^{C}}$$]$ is the uncertainty of the camera optical center
position, initialized to $\epsilon $.$\sigma _{image}$is the image
plane pixel variance, set to 1. $\sigma _{\rho }$is the uncertainty of
the inverse distance. Because the filter mainly deals with distance
feature, $ \sigma _{\rho }$ is initialized to 0.1 to cover any
distance from 50 meters to infinity.

$J$ in equation\ref{equation?} is the Jacobian matrix for 
the initialization equation. 

\begin{equation}
J=\begin{bmatrix}
 & & & & & &0\\
 & &I& & & &\vdots\\ 
 & & & & & &0\\
\frac{\partial p_{i}}{\partial OX_{W}^{C}} &
\frac{\partial p_{i}}{\partial c^{C}} & 
\frac{\partial p_{i}}{\partial r^{C}} & 
0 & \ldots & 0 & 
\frac{\partial p_{i}}{\partial g_{i}} 
\end{bmatrix}
\end{equation}

Whenever a new feature is added, it's initial position is always $
\begin{bmatrix}0&0&0\end{bmatrix}$ in the camera centric coordinate,
and the $\begin{bmatrix}\rho&\varphi&\theta\end{bmatrix}$ parameters
are not a function of $ OX_{W}^{C}$, $c^{C}$,or$r^{C}$, therefore

\begin{equation}
\frac{\partial p_{i}}{\partial OX_{W}^{C}}=0_{6\times 6}
\end{equation}

\begin{equation}
\frac{\partial p_{i}}{\partial c^{C}}=0_{6\times 3}\frac{\partial 
p_{i}}{\partial r^{C}}= 0_{6\times 3}
\end{equation}

$J$ can then be simplified as

\begin{equation}
J=\begin{bmatrix}
I & 0 & \\
0 & \frac{\partial p_{i}}{\partial g_{i}} & \\
\end{bmatrix} 
\end{equation}


\noindent where$g_{i}$ includes the variable in matrix $R$: $
g_{i}=[\begin{matrix}
x_{i}^{C} & y_{i}^{C} & z_{i}^{C} & \rho _{i} & u_{i} & v_{i}
\end{matrix}
]$. Then 

\begin{equation}
\frac{\partial p_{i}}{\partial g_{i}}=\lbrack \begin{bmatrix}
I_{3\times 3} & 0_{3\times 3} & \\
0_{3\times 3} & \begin{bmatrix}
\frac{\partial \rho _{i}}{\partial \rho _{i}} & \frac{\partial \rho 
_{i}}{\partial u_{i}} & \frac{\partial \rho _{i}}{\partial v_{i}} & \\
\frac{\partial \varphi _{i}^{C}}{\partial \rho _{i}} & \frac{\partial 
\varphi _{i}^{C}}{\partial u_{i}} & \frac{\partial \varphi 
_{i}^{C}}{\partial v_{i}} & \\
\frac{\partial \theta _{i}^{C}}{\partial \rho _{i}} & \frac{\partial 
\theta _{i}^{C}}{\partial u_{i}} & \frac{\partial \theta 
_{i}^{C}}{\partial v_{i}} & \\
\end{bmatrix}
 & \\
\end{bmatrix}
\rbrack =\lbrack \begin{bmatrix}
I_{3\times 3} & 0_{3\times 3} & \\
0_{3\times 3} & \begin{bmatrix}
1 & 0 & 0 & \\
0 & \frac{\partial \varphi _{i}^{C}}{\partial u_{i}} & \frac{\partial 
\varphi _{i}^{C}}{\partial v_{i}} & \\
0 & \frac{\partial \theta _{i}^{C}}{\partial u_{i}} & \frac{\partial 
\theta _{i}^{C}}{\partial v_{i}} & \\
\end{bmatrix}
 & \\
\end{bmatrix}
\rbrack $\\
\end{equation}

%%% Local Variables:
%%% mode: latex
%%% TeX-master: "thesis"
%%% End:
