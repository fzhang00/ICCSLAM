\chapter{Algorithm Description}

The algorithm described in Chapter!!!Unresolved reference!!!3 was 
implemented in Python programming language. An open source machine 
vision library OpenCV was utilized to perform feature extraction and 
tracking. The feature extraction method used was the Shi-Tomasi corner 
detector. Feature tracking was accomplished through the pyramid 
implementation of Lucas-Kanade optical flow method . 

\section{Camera Centric Inverse Depth Parameterization}
The standard way of describing a feature's position is to use the 
Euclidean XYZ parameterization. In practical outdoor range estimation 
problem, the algorithm must deal with features located at near infinity. 
Inverse depth parameterization overcomes this problem by representing 
range $d$ in its inverse form $\rho =1/d$. In addition, features at 
infinity contribute in estimating the camera rotational motion even 
though they offer little information on camera translational motion. 
Furthermore, inverse depth parameterization allows us to initialize 
features into the EKF framework before it is safely triangulated.

The inverse depth parameterization used in this work was first 
introduced in . All features and camera positions are referred to a 
world reference frame. When used in an Extended Kalman Filter framework, 
the system suffers decreasing linearity when camera is moving away from 
the world origin. A modified method which uses the camera center as the 
origin was proposed in . Our work has adopted the camera centered 
approach with minor modification to integrate the inertial measurements.

\begin{figure}[h]
\centering
\includegraphics[width=250.55pt,height=109.45pt]{media/image2.eps}
\caption{Figure 2 Inverse Parameterization}
\label{fig:algo1}
\end{figure}

A scene point $p_{i}^{C}$ can be defined by 6 parameters, with the 
superscript $C$ representing a camera reference frame. ():

\begin{equation}
$p_{i}^{C}=\lbrack \begin{matrix}
x_{i}^{C} & y_{i}^{C} & z_{i}^{C} & \rho _{i} & \varphi _{i}^{C} & 
\theta _{i}^{C} & \\
\end{matrix}
\rbrack $\\
 & \\
\end{equation}

The first three parameters $[x_{i}^{C}, y_{i}^{C}, z_{i}^{C}]$ 
represent the initial position where the feature is first observed. $\rho _{i}$ is the inverse distance from the initialization position to 
the feature. The elevation-azimuth pair $[\phi_{i}^{C}, \theta_{i}^{C}]$ encodes a unit vector pointing from the initialization point 
to the feature. The vector is given by

\begin{equation}
$m(\phi_{i}^{C}, \theta_{i}^{C})=\lbrack \begin{matrix}
cos\varphi _{i}^{C}\cos \theta _{i}^{C} & \\
\cos \varphi _{i}^{C}\sin \theta _{i}^{C} & \\
\sin \varphi _{i}^{C} & \\
\end{matrix}
\rbrack $\\
 & \\
\end{equation}


\section{Modeling the System with Extended Kalman 
Filter}

\subsection{Full State Vector}

The EKF state vector is defined as 

\begin{equation}
$x=\lbrack \begin{matrix}
OX_{W}^{C} & c^{C} & r^{C} & p_{1}^{C} & p_{2}^{C} & \ldots & \\
\end{matrix}
\rbrack ^{T}$\\
 & \\
\end{equation}

where $OX_{W}^{C}=\lbrack \begin{matrix}
O_{x}^{C} & O_{y}^{C} & O_{z}^{C} & \\
\end{matrix}
\begin{matrix}
W_{x}^{C} & W_{y}^{C} & W_{z}^{C} & \\
\end{matrix}
\rbrack ^{T}$contains translation parameters $O_{x,y,z}^{C}$ and 
rotation parameters $W_{x,y,z}^{C}$ to transform the camera reference 
frame to the world reference frame. $\lbrack c^{C},r^{C}\rbrack ^{T}$
represents the camera translation and rotation motion frame by frame in 
Euclidean coordinates, and $p_{i}^{C}$contains the feature parameters 
as described in the previous section. 

Prediction (Will be updated to a more complete form)

For a prediction step at time $k$, the world frame and features 
parameters are kept unchanged from time $k-1$. The camera parameters 
are updated using the new inertial measurements: velocity $v^{C}$, 
acceleration $a^{C}$, and rate of change in roll/pitch/yaw $w^{C}$
. The camera motion parameters at time $k$ are then

Equation 2



\begin{center}$\lbrack \begin{matrix}
\begin{matrix}
OX_{W}^{C} & \\
c^{C} & \\
r^{C} & \\
\end{matrix}
 & \\
\begin{matrix}
p_{1}^{C} & \\
p_{2}^{C} & \\
\vdots & \\
\end{matrix}
 & \\
\end{matrix}
\rbrack _{k+1}=\lbrack \begin{matrix}
\begin{matrix}
OX_{W}^{C} & \\
c_{measured}^{C} & \\
r_{measured}^{C} & \\
\end{matrix}
 & \\
\begin{matrix}
p_{1}^{C} & \\
p_{2}^{C} & \\
\vdots & \\
\end{matrix}
 & \\
\end{matrix}
\rbrack $\textbf{ or }$ \lbrack \begin{matrix}
\begin{matrix}
OX_{W}^{C} & \\
c^{C} & \\
r^{C} & \\
\end{matrix}
 & \\
\begin{matrix}
p_{1}^{C} & \\
p_{2}^{C} & \\
\vdots & \\
\end{matrix}
 & \\
\end{matrix}
\rbrack _{k+1}=\lbrack \begin{matrix}
\begin{matrix}
OX_{W}^{C} & \\
c^{C} & \\
r^{C} & \\
\end{matrix}
 & \\
\begin{matrix}
p_{1}^{C} & \\
p_{2}^{C} & \\
\vdots & \\
\end{matrix}
 & \\
\end{matrix}
\rbrack _{k}$\end{center}

Where 

\begin{table}[h]
\centering
\begin{tabular}{|l|l|}
\hline
$c_{k}^{C}=v^{C}\Delta t+ \frac{1}{2}a^{C}\Delta t^{2}$\\
$r_{k}^{C}=r_{k-1}^{C}+ w^{C}$\\
 & \\
\hline
\end{tabular}
\end{table}
Writing the above in the form of $x_{k}= F_{k-1}x_{k-1}+ 
B_{k}u_{k}+w_{k-1}$ , without INS measurement, $F_{k}$\textbf{ }is 
simply\textbf{ }$I$. With INS input, 



$x_{k}=\lbrack \begin{matrix}
\begin{matrix}
I_{6\times 6} & & & \\
 & 0_{3\times 3} & & \\
 & & 0_{3\times 3} & \\
\end{matrix}
 & 0 & \\
0 & \begin{matrix}
I_{6\times 6} & & & \\
 & \ddots & & \\
 & & I_{6\times 6} & \\
\end{matrix}
 & \\
\end{matrix}
\rbrack \lbrack \begin{matrix}
\begin{matrix}
OX_{W}^{C} & \\
c^{C} & \\
r^{C} & \\
\end{matrix}
 & \\
\begin{matrix}
p_{1}^{C} & \\
\vdots & \\
p_{n}^{C} & \\
\end{matrix}
 & \\
\end{matrix}
\rbrack _{k-1}+\lbrack \begin{matrix}
0_{6\times 9 } & \\
\begin{matrix}
\begin{matrix}
1 & 0 & 0 & \\
0 & 1 & 0 & \\
0 & 0 & 1 & \\
\end{matrix}
 & \begin{matrix}
1 & 0 & 0 & \\
0 & 1 & 0 & \\
0 & 0 & 1 & \\
\end{matrix}
 & 0_{3\times 3} & \\
0_{3\times 3} & 0_{3\times 3} & \begin{matrix}
1 & 0 & 0 & \\
0 & 1 & 0 & \\
0 & 0 & 1 & \\
\end{matrix}
 & \\
\end{matrix}
 & \\
0_{6n\times 9} & \\
\end{matrix}
\rbrack \lbrack \begin{matrix}
\begin{matrix}
v_{x}^{C}\Delta t & \\
v_{y}^{C}\Delta t & \\
v_{z}^{C}\Delta t & \\
\end{matrix}
 & \\
\begin{matrix}
\frac{1}{2}a_{x}^{C}\Delta t^{2} & \\
\frac{1}{2}a_{y}^{C}\Delta t^{2} & \\
\frac{1}{2}a_{z}^{C}\Delta t^{2} & \\
\end{matrix}
 & \\
\begin{matrix}
w_{x}^{C} & \\
w_{y}^{C} & \\
w_{z}^{C} & \\
\end{matrix}
 & \\
\end{matrix}
\rbrack +w_{k-1}$\\


Measurement Model

Each observed feature is related to the camera motion through the 
measurement model (). This relationship enables a correction on the 
camera motion and features parameters based on the features' locations 
observed in the image. 

For a feature $p_{i}^{C}$, the vector $h^{R}$pointing from the 
predicted camera location to the feature initialization position is 

\begin{table}[h]
\centering
\begin{tabular}{|l|l|}
\hline
$h_{k}^{R}=\lbrack \begin{matrix}
x_{i}^{C} & \\
y_{i}^{C} & \\
z_{i}^{C} & \\
\end{matrix}
\rbrack _{k}-\lbrack \begin{matrix}
c_{x}^{C} & \\
c_{y}^{C} & \\
c_{z}^{C} & \\
\end{matrix}
\rbrack _{k}$\\
 & Equation 3 \\
\hline
\end{tabular}
\end{table}
The normalized vector pointing from the predicted camera position to the 
feature at time k is then 

\begin{table}[h]
\centering
\begin{tabular}{|l|l|}
\hline
$h_{ k}^{C}=Q^{-1}(r_{k}^{C})(\rho _{k}h_{k}^{R}+m(\varphi _{ 
k}^{C},\theta _{k}^{C}))$\\
 & Equation 4 \\
\hline
\end{tabular}
\end{table}
where$Q^{-1}(r_{k}^{C})$ is the inverse rotation matrix from the 
camera frame at time $k-1$ to camera frame at time $k$. From vector 
$h_{ k}^{C}$, the feature location on image plane can be found by

\begin{table}[h]
\centering
\begin{tabular}{|l|l|}
\hline
$h_{k}^{U}= \lbrack \begin{matrix}
u_{k} & \\
v_{k} & \\
\end{matrix}
\rbrack =\lbrack \begin{matrix}
\frac{s_{x}h_{y,k}^{C}}{h_{x,k}^{C}} & \\
\frac{s_{y}h_{z,k}^{C}}{h_{x,k}^{C}} & \\
\end{matrix}
\rbrack $\\
 & \\
\hline
\end{tabular}
\end{table}
Equation 5

where$s_{x}$ and $s_{y}$ is the scaling factor of the projection, 
obtained through camera calibration.

\begin{figure}[h]
\centering
\includegraphics[width=252.3pt,height=157.25pt]{media/image3.eps}
\end{figure}


\begin{center}Figure 3 Measurement Model\end{center}





Composition Step (will include more detail formulas)

Update step corrects the camera motion and feature location in camera 
frame k-1. To continue to the next cycle of tracking, all parameter must 
be transform to camera frame k. World reference point coordinate and 
orientation from k-1 to k is related by

\begin{center}$\lbrack \begin{matrix}
O_{x}^{C_{k}} & \\
O_{y}^{C} & \\
O_{z}^{C} & \\
\end{matrix}
\rbrack _{k}=R^{-1}(r_{k}^{C_{k-1}})(\lbrack \begin{matrix}
O_{x}^{C_{k-1}} & \\
O_{y}^{C_{k-1}} & \\
O_{z}^{C_{k-1}} & \\
\end{matrix}
\rbrack _{k}- \lbrack \begin{matrix}
c_{x}^{C_{k-1}} & \\
c_{y}^{C_{k-1}} & \\
c_{z}^{C_{k-1}} & \\
\end{matrix}
\rbrack _{k})$\\
\end{center}

Equation 6

$\lbrack \begin{matrix}
 W_{x}^{C_{k}} & \\
W_{y}^{C_{k}} & \\
W_{z}^{C_{k}} & \\
\end{matrix}
\rbrack _{k-1}= \lbrack \begin{matrix}
 W_{x}^{C_{k-1}} & \\
W_{y}^{C_{k-1}} & \\
W_{z}^{C_{k-1}} & \\
\end{matrix}
\rbrack _{k}-r^{C_{k-1}}$\\


Equation 7

Feature parameters in new camera frame are related to the previous frame 
by

$\lbrack \begin{matrix}
x_{i}^{C_{k}} & \\
y_{i}^{C_{k}} & \\
z_{i}^{C_{k}} & \\
\end{matrix}
\rbrack _{k}=Q^{-1}(r^{C_{k-1}})(\lbrack \begin{matrix}
x_{i}^{C_{k-1}} & \\
y_{i}^{C_{k-1}} & \\
z_{i}^{C_{k-1}} & \\
\end{matrix}
\rbrack _{k}- \lbrack \begin{matrix}
c_{x}^{C_{k-1}} & \\
c_{y}^{C_{k-1}} & \\
c_{z}^{C_{k-1}} & \\
\end{matrix}
\rbrack _{k})$\\


Equation 8

\begin{center}$\lbrack \begin{matrix}
\rho _{i} & \\
\varphi _{i}^{C_{k}} & \\
\theta _{i}^{C_{k}} & \\
\end{matrix}
\rbrack _{k}=(\begin{matrix}
\rho _{i, k} & \\
m^{-1}(R^{-1}(r^{C_{k-1}})m(\varphi _{i, k}^{C_{k-1}}, \theta _{i, 
k}^{C_{k-1}}) & \\
\end{matrix}
)$\\
\end{center}

Equation 9

where$m(\varphi _{i, k}^{C_{k-1}}, \theta _{i, k}^{C_{k-1}})$ is the 
unit vector pointing from the initialization point to the feature seen 
by the camera at step k-1

The covariance matrix is also affected by this transform. Therefore must 
be updated. The new covariance matrix is related to the old one by 

$P_{k}^{C_{k}}=J_{C_{k-1}\to C_{k}}P_{k}^{C_{k-1}}J_{C_{k-1}\to 
C_{k}}^{T}$\\


Equation 10

The calculation of $J_{C_{k-1}\to C_{k}}$ is the same as the 
linearization of prediction matrix in section Method 2.

In order to apply the correction and update the camera reference frame 
to the new camera position, an additional composition step is necessary. 
The world reference frame parameters and features parameters are updated 
by applying reference frame transformation from the camera location at 
time k-1 to camera location at time k. The EKF covariance matrix $P_{k}
$ is also updated through

\begin{table}[h]
\centering
\begin{tabular}{|l|l|}
\hline
$P_{k}=JP_{k}J^{T}$\\
 & (8) \\
\hline
\end{tabular}
\end{table}
where$J$ is the Jacobian of the composition equations. 

\subsubsection{Initialization}\label{section:_Toc332876137}
Initialize the State Vector

State vectors are initialized at the first frame. The world origin 
coordinate and orientation, camera motions, and the feature reference 
points are all initialized to zeros, with variance equals to the 
smallest machine number. Thus, 

$OX_{W}^{C}=\lbrack 0 0 0 0 0 0\rbrack $\\


Equation 11

$c^{C}=[0 0 0]$\\


Equation 12

$r^{C}=[0 0 0]$\\


Equation 13

$p_{i}^{C}=[0 0 0 \rho _{i}\varphi _{i}\theta _{i} ]$\\


Equation 14

The inverse distance $\rho $ of all features are initialized to 0.1 
because we are dealing with long distance object. The features 
elevation-azimuth pair $[\varphi _{i}^{C}, \theta _{i}^{C}]$ is 
extracted from features coordinates in image plane. First, a vector 
pointing from camera optical center to feature can be defined by

$h^{C}=\lbrack \begin{matrix}
h_{x}^{C} & \\
h_{y}^{C} & \\
h_{z}^{C} & \\
\end{matrix}
\rbrack =\lbrack \begin{matrix}
1 & \\
u\cdot s_{x} & \\
v\cdot s_{y} & \\
\end{matrix}
\rbrack $\\


Equation 15

Where $[u v]$ is the feature coordinate in the image, and $
[s_{x}s_{y}] $is the scaling factor of the projection from the scene to 
image plane. The elevation-azimuth pair $[\varphi _{i}^{C}, \theta 
_{i}^{C}]$ can then be directly calculated from $h^{C}$

\begin{center}$\varphi 
=arctan(\frac{h_{z}^{C}}{\sqrt{h_{x}^{C}^{2}+h_{y}^{C}^{2}}})$\\
\end{center}

Equation 16

$\theta =arctan(\frac{h_{y}^{C}}{h_{x}^{C}})$\\


Equation 17

Initialized the State Covariance Matrix

Because the world origin is defined at the first frame, it enables 
initializing the filter with minimum variance, which helps reducing the 
lower bound of the filter error. The covariance matrix of the world 
coordinate and orientation, and the camera motion is 

$P=I_{12\times 12}\cdot \epsilon $\\


Equation 18

where$I$ is a 12-by-12 identity matrix, and $\epsilon $ is the 
lowest significant bit (LSB) of a machine.

\ \ \ \ The covariance of features is added one by one as there is 
correlation between them. For every new feature added, the new 
covariance matrix becomes

$P_{new}=J\lbrack \begin{matrix}
P_{old} & 0 & \\
0 & R & \\
\end{matrix}
\rbrack J^{T}$\\


Equation 19

where$P_{old}$ isthe covariance matrix of the existing state vector, 
and the initial $P_{old}$is defined in . Matrix R is the covariance 
matrix of the variable in features initialization - .

$R=\lbrack \begin{matrix}
\sigma _{x_{i}^{C}} & & & & & & \\
 & \sigma _{y_{i}^{C}} & & & 0 & & \\
 & & \sigma _{z_{i}^{C}} & & & & \\
 & & & \sigma _{\rho } & & & \\
 & 0 & & & \sigma _{image} & & \\
 & & & & & \sigma _{image} & \\
\end{matrix}
\rbrack =\lbrack \begin{matrix}
\epsilon & & & & & & \\
 & \epsilon & & & 0 & & \\
 & & \epsilon & & & & \\
 & & & 0.1 & & & \\
 & 0 & & & 1 & & \\
 & & & & & 1 & \\
\end{matrix}
\rbrack $\\


Equation 20

where$[\sigma _{x_{i}^{C}}\sigma _{y_{i}^{C}}\sigma _{z_{i}^{C}}$$]$ 
is the uncertainty of the camera optical center position, initialized to 
$\epsilon $.$\sigma _{image}$is the image plane pixel variance, 
set to 1. $\sigma _{\rho }$is the uncertainty of the inverse 
distance. Because the filter mainly deals with distance feature, $
\sigma _{\rho }$ is initialized to 0.1 to cover any distance from 50 
meters to infinity. 

$J$in!!!Unresolved reference!!!Equation 1 is the Jacobian matrix for 
the initialization equation. 

$J=\lbrack \begin{matrix}
I & \begin{matrix}
0 & \\
\vdots & \\
0 & \\
\end{matrix}
 & \\
\begin{matrix}
\begin{matrix}
\frac{\partial p_{i}}{\partial OX_{W}^{C}} & \frac{\partial 
p_{i}}{\partial c^{C}} & \frac{\partial p_{i}}{\partial r^{C}} & \\
\end{matrix}
 & \begin{matrix}
0 & \ldots & 0 & \\
\end{matrix}
 & \\
\end{matrix}
 & \frac{\partial p_{i}}{\partial g_{i}} & \\
\end{matrix}
\rbrack $\\


Equation 21

Whenever a new feature is added, it's initial position is always $
\lbrack 0 0 0\rbrack $ in the camera centric coordinate, and the $
\lbrack \rho \varphi \theta \rbrack $ parameters are not a function of$
OX_{W}^{C}$, $c^{C}$,or$r^{C}$, therefore

$\frac{\partial p_{i}}{\partial OX_{W}^{C}}=0_{6\times 6}$\\


\begin{center}Equation 22\end{center}

$\frac{\partial p_{i}}{\partial c^{C}}=0_{6\times 3}\frac{\partial 
p_{i}}{\partial r^{C}}= 0_{6\times 3}$\\


\begin{center}Equation 23\end{center}

$J$canthen be simplified as

$J=\lbrack \begin{matrix}
I & 0 & \\
0 & \frac{\partial p_{i}}{\partial g_{i}} & \\
\end{matrix}
\rbrack $\\


\begin{center}Equation 24\end{center}

Where$g_{i}$ includes the variable in matrix $R$: $
g_{i}=[\begin{matrix}
\begin{matrix}
x_{i}^{C} & y_{i}^{C} & z_{i}^{C} & \\
\end{matrix}
 & \begin{matrix}
\rho _{i} & u_{i} & v_{i} & \\
\end{matrix}
 & \\
\end{matrix}
]$. Then 

$\frac{\partial p_{i}}{\partial g_{i}}=\lbrack \begin{matrix}
I_{3\times 3} & 0_{3\times 3} & \\
0_{3\times 3} & \begin{matrix}
\frac{\partial \rho _{i}}{\partial \rho _{i}} & \frac{\partial \rho 
_{i}}{\partial u_{i}} & \frac{\partial \rho _{i}}{\partial v_{i}} & \\
\frac{\partial \varphi _{i}^{C}}{\partial \rho _{i}} & \frac{\partial 
\varphi _{i}^{C}}{\partial u_{i}} & \frac{\partial \varphi 
_{i}^{C}}{\partial v_{i}} & \\
\frac{\partial \theta _{i}^{C}}{\partial \rho _{i}} & \frac{\partial 
\theta _{i}^{C}}{\partial u_{i}} & \frac{\partial \theta 
_{i}^{C}}{\partial v_{i}} & \\
\end{matrix}
 & \\
\end{matrix}
\rbrack =\lbrack \begin{matrix}
I_{3\times 3} & 0_{3\times 3} & \\
0_{3\times 3} & \begin{matrix}
1 & 0 & 0 & \\
0 & \frac{\partial \varphi _{i}^{C}}{\partial u_{i}} & \frac{\partial 
\varphi _{i}^{C}}{\partial v_{i}} & \\
0 & \frac{\partial \theta _{i}^{C}}{\partial u_{i}} & \frac{\partial 
\theta _{i}^{C}}{\partial v_{i}} & \\
\end{matrix}
 & \\
\end{matrix}
\rbrack $\\


Equation 25

Based on the rule of derivation, 

$\frac{\partial \varphi _{i}^{C}}{\partial u_{i}}=\frac{\partial 
\varphi _{i}^{C}}{\partial h^{C}}\frac{\partial h^{C}}{\partial u_{i}}$
\\


$\frac{\partial \theta _{i}^{C}}{\partial u_{i}}=\frac{\partial \theta 
_{i}^{C}}{\partial h^{C}}\frac{\partial h^{C}}{\partial u_{i}}$\\


$\frac{\partial \varphi _{i}^{C}}{\partial v_{i}}=\frac{\partial 
\varphi _{i}^{C}}{\partial h^{C}}\frac{\partial h^{C}}{\partial v_{i}}$
\\


$\frac{\partial \theta _{i}^{C}}{\partial v_{i}}=\frac{\partial \theta 
_{i}^{C}}{\partial h^{C}}\frac{\partial h^{C}}{\partial v_{i}}$\\


Equation 26

and

$\frac{\partial \varphi _{i}^{C}}{\partial h^{C}}=\lbrack 
\begin{matrix}
\frac{\partial \varphi _{i}^{C}}{\partial h_{x}^{C}} & \frac{\partial 
\varphi _{i}^{C}}{\partial h_{y}^{C}} & \frac{\partial \varphi 
_{i}^{C}}{\partial h_{z}^{C}} & \\
\end{matrix}
\rbrack =\lbrack \begin{matrix}
\frac{-h_{x}^{C}\cdot 
h_{z}^{C}}{(h_{x}^{C}^{2}+h_{y}^{C}^{2}+h_{z}^{C}^{2})\sqrt{h_{x}^{C}^{2}+h_{y}^{C}^{2}}} 
& \\
\frac{-h_{y}^{W}\cdot 
h_{z}^{W}}{(h_{x}^{C}^{2}+h_{y}^{C}^{2}+h_{z}^{C}^{2})\sqrt{h_{x}^{C}^{2}+h_{y}^{C}^{2}}} 
& \\

\frac{\sqrt{h_{x}^{C}^{2}+h_{y}^{C}^{2}}}{(h_{x}^{C}^{2}+h_{y}^{C}^{2}+h_{z}^{C}^{2})} 
& \\
\end{matrix}
\rbrack ^{T}$\\


Equation 27

$\frac{\partial \theta _{i}^{C}}{\partial h^{C}}=\lbrack \begin{matrix}
\frac{\partial \theta _{i}^{C}}{\partial h_{x}^{C}} & \frac{\partial 
\theta _{i}^{C}}{\partial h_{y}^{C}} & \frac{\partial \theta 
_{i}^{C}}{\partial h_{z}^{C}} & \\
\end{matrix}
\rbrack =\lbrack \begin{matrix}
-\frac{h_{y}^{C}}{h_{x}^{C}^{2}+h_{y}^{C}^{2}} & 
\frac{h_{x}^{C}}{h_{x}^{C}^{2}+h_{y}^{C}^{2}} & 0 & \\
\end{matrix}
\rbrack $\\


Equation 28

\ \ \ \ 

$\frac{\partial h^{C}}{\partial u_{i}}=\lbrack \begin{matrix}
\frac{\partial h_{x}^{C}}{\partial u_{i}} & \\
\frac{\partial h_{y}^{C}}{\partial u_{i}} & \\
\frac{\partial h_{z}^{C}}{\partial u_{i}} & \\
\end{matrix}
\rbrack =\lbrack \begin{matrix}
0 & \\
s_{x} & \\
0 & \\
\end{matrix}
\rbrack $\\


Equation 29



$\frac{\partial h^{C}}{\partial v_{i}}=\lbrack \begin{matrix}
\frac{\partial h_{x}^{C}}{\partial v_{i}} & \\
\frac{\partial h_{y}^{C}}{\partial v_{i}} & \\
\frac{\partial h_{z}^{C}}{\partial v_{i}} & \\
\end{matrix}
\rbrack =\lbrack \begin{matrix}
0 & \\
0 & \\
s_{y} & \\
\end{matrix}
\rbrack $\\


Equation 30



\subsubsection{Linearization of Measurement Model}
The measurement model of the algorithm is not linear, and must be 
linearized by evaluating first derivative of the measurement equation at 
the predicted camera location. Since the measurement model is not a 
function of world origin coordinate and orientation, the Jacobian matrix 
of the measurement model is\ \ \ \ 

$\frac{\partial h_{k}^{U}(x)}{\partial x}=\lbrack \begin{matrix}
\begin{matrix}
0_{2n\times 6} & \frac{\partial h_{k}^{U}(x)}{\partial c^{C}} & 
\frac{\partial h_{k}^{U}(x)}{\partial r^{C}} & \\
\end{matrix}
 & \begin{matrix}
\frac{\partial h_{k}^{U}(x)}{\partial p_{1}} & \ldots & \frac{\partial 
h_{k}^{U}(x)}{\partial p_{n}} & \\
\end{matrix}
 & \\
\end{matrix}
\rbrack =\lbrack \begin{matrix}
\begin{matrix}
0_{2n\times 6} & \frac{\partial h_{k}^{U}}{\partial h_{k}^{C}}\cdot 
\frac{\partial h_{k}^{C}}{\partial c^{C}} & \frac{\partial 
h_{k}^{U}}{\partial h_{k}^{C}}\cdot \frac{\partial h_{k}^{C}}{\partial 
r^{C}} & \\
\end{matrix}
 & \begin{matrix}
\frac{\partial h_{k}^{U}}{\partial h_{k}^{C}}\cdot \frac{\partial 
h_{k}^{C}}{\partial p_{1}} & \ldots & \frac{\partial h_{k}^{U}}{\partial 
h_{k}^{C}}\cdot \frac{\partial h_{k}^{C}}{\partial p_{n}} & \\
\end{matrix}
 & \\
\end{matrix}
\rbrack $\\


Equation 31

Elements in the matrix above are given in - . The formula for 
calculating $Q^{-1}$in - can be found in 

\begin{center}$\frac{\partial h_{k}^{U}}{\partial h_{ k}^{C}}= \lbrack 
\begin{matrix}
-\frac{s_{x}h_{y,k}^{C}}{h_{x,k}^{C}^{2}} & \frac{s_{x}}{h_{x,k}^{C}} & 
0 & \\
-\frac{s_{y}h_{z,k}^{C}}{h_{x,k}^{C}^{2}} & 0 & 
\frac{s_{y}}{h_{x,k}^{C}} & \\
\end{matrix}
\rbrack $\\
\end{center}

Equation 32



$\frac{\partial h^{C}}{\partial c^{C}}=-R^{-1}(r_{k}^{C})\rho _{k}$\\


Equation 33

$\frac{\partial h(x)}{\partial r^{C}}=\lbrack \frac{\partial 
h^{C}}{\partial r_{x}^{C}}\frac{\partial h^{C}}{\partial 
r_{y}^{C}}\frac{\partial h^{C}}{\partial r_{z}^{C}}\rbrack $\\


Equation 34

$\frac{\partial h^{C}}{\partial r_{x}}=\lbrack \begin{matrix}
0 & \\

h_{x}^{C}(sin(r_{x})sin(r_{z})+cos(r_{x})cos(r_{z})sin(r_{y}))+h_{y}^{C}(-cos(r_{z})sin(r_{x})+cos(r_{x})sin(r_{y})sin(r_{z})) 
+ h_{z}^{C}cos(r_{x})cos(r_{y}) & \\
h_{x}^{C}(cos(r_{x})\sin 
(r_{z})-cos(r_{z})sin(r_{x})sin(r_{y}))+h_{y}^{C}(-cos(r_{x})cos(r_{z})-sin(r_{x})sin(r_{y})sin(r_{z})) 
- h_{z}^{C}cos(r_{y})sin(r_{x}) & \\
\end{matrix}
\rbrack $\\


Equation 35

$\frac{\partial h^{C}}{\partial r_{y}}=\lbrack \begin{matrix}
-h_{z}^{C}\cos (r_{y})- h_{x}^{C}\cos (r_{z})sin(r_{y})- 
h_{y}^{C}sin(r_{y})sin(r_{z}) & \\

-h_{z}^{C}sin(r_{x})sin(r_{y})+h_{x}^{C}cos(r_{y})cos(r_{z})sin(r_{x})+h_{y}^{C}cos(r_{y})sin(r_{x})sin(r_{z}) 
& \\

-h_{z}^{C}cos(r_{x})sin(r_{y})+h_{x}^{C}cos(r_{x})cos(r_{y})cos(r_{z})+h_{y}^{C}cos(r_{x})cos(r_{y})sin(r_{z}) 
& \\
\end{matrix}
\rbrack $\\


Equation 36

$\frac{\partial h^{C}}{\partial r_{z}}=\lbrack \begin{matrix}
h_{y}^{C}cos(r_{y})cos(r_{z})-h_{x}^{C}cos(r_{y})sin(r_{z}) & \\
h_{x}^{C}(-cos(r_{x})\cos (r_{z})- sin(r_{x})\sin (r_{y})\sin (r_{z}))+ 
h_{y}^{C}(-cos(r_{x})sin(r_{z})+cos(r_{z})sin(r_{x})sin(r_{y})) & \\

h_{x}^{C}(cos(r_{z})sin(r_{x})-cos(r_{x})sin(r_{y})sin(r_{z}))+h_{y}^{C}(sin(r_{x})sin(r_{z})+cos(r_{x})cos(r_{z})sin(r_{y})) 
& \\
\end{matrix}
\rbrack $\\


Equation 37



\begin{center}$\frac{\partial h^{C}}{\partial p_{i}}=\lbrack 
\begin{matrix}
\frac{\partial h^{C}}{\partial x_{i}^{C}} & \frac{\partial 
h^{C}}{\partial y_{i}^{C}} & \frac{\partial h^{C}}{\partial z_{i}^{C}} & 
\\
\end{matrix}
\frac{\partial h^{C}}{\partial \rho _{i}}\frac{\partial h^{C}}{\partial 
\theta _{i}}\frac{\partial h^{C}}{\partial \varphi _{i}}\rbrack $\\
\end{center}

Equation 38 is a 3x6 matrix

$\lbrack \begin{matrix}
\frac{\partial h^{C}}{\partial x_{i}^{C}} & \frac{\partial 
h^{C}}{\partial y_{i}^{C}} & \frac{\partial h^{C}}{\partial z_{i}^{C}} & 
\\
\end{matrix}
\rbrack =Q^{-1}(r_{k}^{C})\rho _{k}$\\


Equation 39

$\frac{\partial h^{C}}{\partial \rho _{i}}=-Q^{-1}(r_{k}^{C})\cdot 
c_{k}$\\


Equation 40

$\frac{\partial h^{C}}{\partial \theta _{i}}=Q^{-1}(r_{k}^{C})\lbrack 
\begin{matrix}
-cos(\varphi _{i})\sin (\theta _{i}) & \\
\cos (\varphi _{i})cos?(\theta _{i}) & \\
0 & \\
\end{matrix}
\rbrack $\\


Equation 41

$\frac{\partial h^{C}}{\partial \varphi _{i}}= Q^{-1}(r_{k}^{C})\lbrack 
\begin{matrix}
-cos(\varphi _{i})sin?(\theta _{i}) & \\
-sin(\varphi _{i})sin?(\theta _{i}) & \\
cos?(\varphi _{i}) & \\
\end{matrix}
\rbrack $\\


Equation 42

\subsubsection{Jacobian Matrix of Composition Equations}
$J_{C_{k-1}\to C_{k}}=\frac{\partial x_{k}}{\partial x_{k-1}}=\lbrack 
\frac{\partial x_{k}}{\partial OX_{W, k-1}^{C}}\begin{matrix}
\frac{\partial x_{k}}{\partial c_{k-1}} & \frac{\partial x_{k}}{\partial 
r_{k-1}} & \\
\end{matrix}
\frac{\partial x_{k}}{\partial p_{1, k-1}^{C}}\frac{\partial 
x_{k}}{\partial p_{2, k-1}^{C}} \cdots \rbrack $\\


Equation 43

$\frac{\partial x_{k}}{\partial OX_{W, k-1}^{C}}$is given by:

$\frac{\partial x_{k}}{\partial OX_{W, k-1}^{C}}=\lbrack \begin{matrix}
\frac{\partial O_{W,k}^{C}}{\partial O_{W,k-1}^{C}} & \frac{\partial 
O_{W,k}^{C}}{\partial W_{W,k-1}^{C}} & \\
\begin{matrix}
\frac{\partial W_{W,k}^{C}}{\partial O_{W,k-1}^{C}} & \\
\frac{\partial c_{k}}{\partial O_{W,k-1}^{C}} & \\
\frac{\partial r_{k}}{\partial O_{W,k-1}^{C}} & \\
\end{matrix}
 & \begin{matrix}
\frac{\partial W_{W,k}^{C}}{\partial W_{W,k-1}^{C}} & \\
\frac{\partial c_{k}}{\partial W_{W,k-1}^{C}} & \\
\frac{\partial r_{k}}{\partial W_{W,k-1}^{C}} & \\
\end{matrix}
 & \\
\frac{\partial p_{i}^{C}}{\partial O_{W,k-1}^{C}} & \frac{\partial 
p_{i}^{C}}{\partial W_{W,k-1}^{C}} & \\
\end{matrix}
\rbrack =\lbrack \begin{matrix}
Q^{-1}(r^{C}) & 0 & \\
0 & I_{3\times 3} & \\
0 & 0 & \\
0 & 0 & \\
0 & 0 & \\
\end{matrix}
\rbrack $\\


Equation 44

Derivatives of $x_{k}$with respect to $c$ and $p$are given in - 
. 

$\lbrack \begin{matrix}
\frac{\partial O_{W,k}^{C}}{\partial c_{k-1}} & \frac{\partial 
O_{W,k}^{C}}{\partial r_{k-1}} & \\
\begin{matrix}
\frac{\partial W_{W,k}^{C}}{\partial c_{k-1}} & \\
\frac{\partial c_{k}}{\partial c_{k-1}} & \\
\frac{\partial r_{k}}{\partial c_{k-1}} & \\
\end{matrix}
 & \begin{matrix}
\frac{\partial W_{W,k}^{C}}{\partial r_{k-1}} & \\
\frac{\partial c_{k}}{\partial r_{k-1}} & \\
\frac{\partial r_{k}}{\partial r_{k-1}} & \\
\end{matrix}
 & \\
\frac{\partial p(0:2)_{i}^{C}}{\partial c_{k-1}} & \frac{\partial 
p(0:2)_{i}^{C}}{\partial r_{k-1}} & \\
\frac{\partial p(3:5)_{i}^{C}}{\partial c_{k-1}} & \frac{\partial 
p(3:5)_{i}^{C}}{\partial r_{k-1}} & \\
\end{matrix}
\rbrack =\lbrack \begin{matrix}
-Q^{-1}(r_{k}^{C_{k-1}}) & \frac{\partial 
Q^{-1}(r_{k}^{C_{k-1}})}{\partial r_{k-1}} & \\
0_{3\times 3} & -I_{3\times 3} & \\
I_{3\times 3} & 0_{3\times 3} & \\
0_{3\times 3} & I_{3\times 3} & \\
-Q^{-1}(r_{k}^{C_{k-1}}) & \frac{\partial 
Q^{-1}(r_{k}^{C_{k-1}})}{\partial r_{k-1}} & \\
0_{3\times 3} & pr & \\
\end{matrix}
\rbrack $\\


Equation 45

where

$pr=\lbrack \begin{matrix}
\begin{matrix}
1 & 0 & 0 & \\
\end{matrix}
 & \\
\frac{\partial \varphi _{i, k}^{C}}{\partial r_{k-1}} & \\
\frac{\partial \theta _{i, k}^{C}}{\partial r_{k-1}} & \\
\end{matrix}
\rbrack =\lbrack \begin{matrix}
\begin{matrix}
1 & 0 & 0 & \\
\end{matrix}
 & \\
\frac{\partial m^{-1}(Q^{-1}(r_{k}^{C_{k-1}})m(\varphi _{i, 
k}^{C_{k-1}}, \theta _{i, k}^{C_{k-1}}))}{\partial r_{k-1}} & \\
\end{matrix}
\rbrack $\\


Equation 46

Let $M=Q^{-1}(r_{k}^{C_{k-1}})m(\varphi _{i, k}^{C_{k-1}}, \theta _{i, 
k}^{C_{k-1}})$, and $m^{-1}=m^{-1}(M)$, then

$\frac{\partial m^{-1}}{\partial r_{k}^{C_{k-1}}}=\frac{\partial 
m^{-1}}{\partial M}\cdot \frac{\partial M}{\partial r_{k}^{C_{k-1}}}$
\\


Equation 47

where

$\frac{\partial M}{\partial r_{k}^{C_{k-1}}}= \frac{\partial 
Q^{-1}(r_{k}^{C_{k-1}})}{\partial r_{k-1}}\cdot m(\varphi _{i, 
k}^{C_{k-1}}, \theta _{i, k}^{C_{k-1}})$\\


Equation 48

as$m^{-1}$isthe function that calculate$[\varphi \theta ]$ from a 
vector,$\frac{\partial m^{-1}}{\partial M}$ is the same as and .

Derivatives of $x_{k}$with respect to feature parameters$p_{i}^{C}
$are given in - 

$\frac{\partial x_{k}^{C_{k}}}{\partial p_{i,k}^{C_{k-1 }}}=\lbrack 
\begin{matrix}
\begin{matrix}
\frac{\partial O_{W,k}^{C_{k}}}{\partial p_{i,k}^{C_{k-1 }}} & \\
\frac{\partial W_{W,k}^{C_{k}}}{\partial p_{i,k}^{C_{k-1 }}} & \\
\frac{\partial c_{k}^{C_{k}}}{\partial p_{i,k}^{C_{k-1 }}} & \\
\end{matrix}
 & \\
\begin{matrix}
\frac{\partial r_{k}^{C_{k}}}{\partial p_{i,k}^{C_{k-1 }}} & \\
\frac{\partial p_{i, k}^{C}}{\partial p_{i,k}^{C_{k-1 }}} & \\
\end{matrix}
 & \\
\end{matrix}
\rbrack =\lbrack \begin{matrix}
0 & \\
0 & \\
0 & \\
0 & \\
\frac{\partial p_{i,k}^{C_{k }}}{\partial p_{i,k}^{C_{k-1 }}} & \\
\end{matrix}
\rbrack $\\


Equation 49

$\frac{\partial p_{i,k}^{C_{k }}}{\partial p_{i,k}^{C_{k-1 }}}=\lbrack 
\begin{matrix}
\frac{\partial x_{i,k}^{C_{k}}}{\partial p_{i,k}^{C_{k-1 }}} & \\
\frac{\partial y_{i,k}^{C}}{\partial p_{i,k}^{C_{k-1 }}} & \\
\frac{\partial z_{i,k}^{C}}{\partial p_{i,k}^{C_{k-1 }}} & \\
\frac{\partial \rho _{i,k}^{C}}{\partial p_{i,k}^{C_{k-1 }}} & \\
\frac{\partial \varphi _{i, k}^{C}}{\partial p_{i,k}^{C_{k-1 }}} & \\
\frac{\partial \theta _{i, k}^{C}}{\partial p_{i,k}^{C_{k-1 }}} & \\
\end{matrix}
\rbrack =\lbrack \begin{matrix}
Q^{-1}(r_{k}^{C_{k-1}}) & 0_{3\times 3} & \\
0_{3\times 3} & \frac{\partial p(3:5)_{i,k}^{C_{k}}}{\partial 
p(3:5)_{i,k}^{C_{k-1}}} & \\
\end{matrix}
\rbrack $\\


Equation 50



$\frac{\partial p(3:5)_{i,k}^{C_{k}}}{\partial 
p(3:5)_{i,k}^{C_{k-1}}}=\lbrack \begin{matrix}
1 & 0 & 0 & \\
0 & \frac{\partial \varphi _{i, k}^{C_{k}}}{\partial \varphi _{i, 
k}^{C_{k-1}}} & \frac{\partial \varphi _{i, k}^{C_{k}}}{\partial \theta 
_{i, k}^{C_{k-1}}} & \\
0 & \frac{\partial \theta _{i, k}^{C_{k}}}{\partial \varphi _{i, 
k}^{C_{k-1}}} & \frac{\partial \theta _{i, k}^{C_{k}}}{\partial \theta 
_{i, k}^{C_{k-1}}} & \\
\end{matrix}
\rbrack =\lbrack \begin{matrix}
1 & 0 & 0 & \\
0 & \frac{\partial \varphi _{i, k}^{C_{k}}}{\partial M}\frac{\partial 
M}{\partial \varphi _{i, k}^{C_{k-1}}} & \frac{\partial \varphi _{i, 
k}^{C_{k}}}{\partial M}\frac{\partial M}{\partial \theta _{i, 
k}^{C_{k-1}}} & \\
0 & \frac{\partial \theta _{i, k}^{C_{k}}}{\partial M}\frac{\partial 
M}{\partial \varphi _{i, k}^{C_{k-1}}} & \frac{\partial \theta _{i, 
k}^{C_{k}}}{\partial M}\frac{\partial M}{\partial \theta _{i, 
k}^{C_{k-1}}} & \\
\end{matrix}
\rbrack $\\


Equation 51

$\frac{\partial \varphi _{i, k}^{C_{k}}}{\partial M}$is the same as . 
$\frac{\partial \theta _{i, k}^{C_{k}}}{\partial M}$is the same as . 
$\frac{\partial M}{\partial \varphi _{i,k}^{C_{k-1}}}$and$
\frac{\partial M}{\partial \theta _{i, k}^{C_{k-1}}}$ is given below.

$\frac{\partial M}{\partial \varphi 
_{i,k}^{C_{k-1}}}=R^{-1}(r^{C})\lbrack \begin{matrix}
-\cos (\theta _{i, k}^{C_{k-1}})\sin (\varphi _{i,k}^{C_{k-1}}) & \\
-\sin (\varphi _{i,k}^{C_{k-1}})\sin (\theta _{i, k}^{C_{k-1}}) & \\
\cos (\varphi _{i,k}^{C_{k-1}}) & \\
\end{matrix}
\rbrack $\\


Equation 52

$\frac{\partial M}{\partial \theta _{i, 
k}^{C_{k-1}}}=R^{-1}(r^{C})\lbrack \begin{matrix}
-\cos (\varphi _{i,k}^{C_{k-1}})\sin (\theta _{i, k}^{C_{k-1}}) & \\
\cos (\varphi _{i,k}^{C_{k-1}})\cos (\theta _{i, k}^{C_{k-1}}) & \\
0 & \\
\end{matrix}
\rbrack $\\


Equation 53

\subsubsection{Adding and Deleting 
Features}\label{section:_Toc332876136}
Features that move out of the camera's FOI are removed from the filter. 
The removal is done by directly deleting the parameters from the state 
vector, and the related rows and columns from the state covariance 
matrix. The parameters of the deleted features are still recorded into 
the database, and remained unchanged for all iteration after the removal 
unless a GPS bundle correction occurs. 

To maintain sufficient amount of features to generate map, new features 
are acquired and added to the filter if a feature deletion procedure 
occurred, or number of tracked feature is lower than a threshold. The 
feature addition procedure occurs after the composition step of an 
iteration, and follows the same procedure as described in section . 

\subsubsection{Bundle Correction with GPS (not yet 
implemented)}\label{section:_Toc332876138}
Apply overall correction on the map at a sparser time interval using the 
GPS positions. 

\subsubsection{Processing for Comparison to Ground Truth Data}
The accuracy of the result was analyzed by comparing to the digital 
elevation models (DEM) .To make the comparison, the features coordinate 
and the DEM data must be brought to the same coordinate system. For ease 
of viewing, the world coordinate is used. The features' positions can be 
converted to Euclidean representation in world frame by: 

$\lbrack \begin{matrix}
X_{i}^{W} & \\
Y_{i}^{W} & \\
Z_{i}^{W} & \\
\end{matrix}
\rbrack =Q^{-1}(O_{XYZ}^{c}, W_{XYZ}^{c})(\lbrack \begin{matrix}
x_{i}^{C} & \\
y_{i}^{C} & \\
z_{i}^{C} & \\
\end{matrix}
\rbrack +\frac{1}{\rho _{i}}m(\varphi _{i}^{C}, \theta _{i}^{C}))$\\


The DEM is converted into the world frame using the UAV's initial GPS 
location$[\begin{matrix}
Latitude_{init} & Longitude_{init} & Height_{init} & \\
\end{matrix}
]$ and orientation$\lbrack \begin{matrix}
Roll_{init} & Pitch_{init } & Azimuth_{init} & \\
\end{matrix}
\rbrack $. First, the UAV's GPS latitude and longitude is converted 
into UTM representation$\begin{matrix}
[Northing_{init} & Easting_{init} & Height_{init}] & \\
\end{matrix}
$. Then, a transformation matrix can be defined as followed:

$Q_{Roll}^{-1}=\lbrack \begin{matrix}
1 & 0 & 0 & 0 & \\
0 & \cos (Roll_{init}) & \sin (Roll_{init}) & 0 & \\
0 & -sin(Roll_{init}) & \cos (Roll_{init}) & 0 & \\
0 & 0 & 0 & 1 & \\
\end{matrix}
\rbrack $\\


$Q_{Pitch}^{-1}=\lbrack \begin{matrix}
\cos (Pitch_{init}+\pi ) & 0 & -\sin (Pitcch_{init}+\pi ) & 0 & \\
0 & 1 & 0 & 0 & \\
\sin (Pitch_{init}+\pi ) & 0 & \cos (Pitch_{init}+\pi ) & 0 & \\
0 & 0 & 0 & 1 & \\
\end{matrix}
\rbrack $\\


$Q_{Azimuth}^{-1}=\lbrack \begin{matrix}
\cos (Azimuth_{init}+\frac{\pi }{2}) & \sin (Azimuth_{init}+\frac{\pi 
}{2}) & 0 & 0 & \\
-sin(Azimuth_{init}+\frac{\pi }{2}) & \cos (Azimuth_{init}+\frac{\pi 
}{2}) & 0 & 0 & \\
0 & 0 & 1 & 0 & \\
0 & 0 & 0 & 1 & \\
\end{matrix}
\rbrack $\\


$Q_{T}^{-1}=\lbrack \begin{matrix}
1 & 0 & 0 & -Northing_{init} & \\
0 & 1 & 0 & -Easting_{init} & \\
0 & 0 & 1 & -Height_{init} & \\
0 & 0 & 0 & 1 & \\
\end{matrix}
\rbrack $\\


$Q^{-1}=Q_{T}^{-1}\cdot Q_{Roll}^{-1}\cdot Q_{Pitch}^{-1}\cdot 
Q_{Azimuth}^{-1}$\\


At last, the DEM in world coordinate is given by

$DEM_{World}=Q^{-1}\cdot DEM_{UTM}$\\
