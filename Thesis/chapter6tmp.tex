



\begin{figure}[h]
\centering
\includegraphics[width=15.24cm,height=5.30cm]{media/image59.eps}
\end{figure}


\begin{figure}[h]
\centering
\includegraphics[width=15.24cm,height=5.36cm]{media/image60.eps}
\end{figure}


\begin{figure}[h]
\centering
\includegraphics[width=15.24cm,height=8.85cm]{media/image61.eps}
\end{figure}





\begin{figure}[h]
\centering
\includegraphics[width=15.24cm,height=9.67cm]{media/image62.eps}
\end{figure}


\begin{figure}[h]
\centering
\includegraphics[width=15.24cm,height=9.66cm]{media/image63.eps}
\caption{\label{figure:_Ref349492417} FY offset vs. estimated feature 
coordinate error}
\end{figure}



 and show the features coordinates error for all the $ f_{x}, f_{y}$ 
setting. The following characters can be observed from the plot:

\begin{itemize}
\item Deviation of fx has more affect feature position estimation on the 
Y and X axis; deviation of fy from its true value affect feature 
position estimation on Z and X axis.
\item Deviation of fx or fy on either direction causes X component of 
the estimated feature position to be smaller than the truth value. 
\item Negative deviation of fx causes little mean error on Y, (fy on Z), 
but the mean squared error is equivalent to the amount caused by the 
positive deviation. This indicates that an offset on fx to negative side 
is not any better than to the positive side. 
\end{itemize}
\begin{figure}[h]
\centering
\includegraphics[width=15.24cm,height=5.22cm]{media/image64.eps}
\end{figure}


\begin{figure}[h]
\centering
\includegraphics[width=15.24cm,height=8.55cm]{media/image65.eps}
\end{figure}


\begin{figure}[h]
\centering
\includegraphics[width=15.24cm,height=7.40cm]{media/image66.eps}
\caption{\label{figure:_Ref349768778}}
\end{figure}



 shows the estimated feature position error versus the camera distortion. 
Since in CC\_SLAM no camera distortion is included, the true value of 
the distortion is $[$0, 0, 0, 0$]$. The calibrated distortion parameters 
are marked at 100\% deviation from true value. Therefore, the simulation 
results at 100\% offset shows the average error of the estimated feature 
position resulted from the neglecting of the distortion factor in the 
actual flight result. From , it can be found that lens distortion impact 
on the X axis position error the most. At 100\%, the estimated value is 
50 meters closer than true value on average. 

Secondly, lens distortion has more impact on features located far from 
the center of the image. shows feature position error vs projected 
feature distance from the optical center on image. Feature position 
error on X axis increases when feature lies further from the image 
center. On axis Y and Z, there is no obvious correlation between 
position error and feature distance from image center. 

\begin{figure}[h]
\centering
\includegraphics[width=15.71cm,height=5.41cm]{media/image67.eps}
\end{figure}


Figure 40 Estimated feature position error vs. feature distance on from 
image center on image plane. 

 

\begin{figure}[h]
\centering
\includegraphics[width=12.18cm,height=9.14cm]{media/image68.eps}
\end{figure}
