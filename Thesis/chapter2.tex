\chapter{REVIEW ON SENSORS AND RELATED WORK}\label{ch:Review}

\section{SENSORS FOR OBSTACLE DETECTION}\label{sec:sensor}

\subsection{Overview}\label{sec:SensorOverview}

Many work in obstacle detection uses range sensors such as radar, laser 
range finder, LIDAR, sonar. $[$reference goes here$]$. Radar and laser 
range finder provides only point measurement at a given position and 
orientation. To acquire a full 3D range map of a scene, mechanical 
scanning mechanisms are required, which limits the data acquisition rate 
of these device. LIDAR operate in the same manner as laser range finder, 
except with the scanning mechanism built in. These sensors usually have 
high power requirement and mass, and may not be suitable for small and 
mid size UAV. Sonar is usually used in indoor or under water 
applications, and have wide beam profile which make it difficult to 
identify the origin of return signal, and results in low resolution 
range map. 3D flash LIDAR is capable of acquire 3D range measurement 
simultaneously by illuminating the entire field of view of the camera 
with a single laser, and capturing the reflected laser with a 2D imaging 
sensor $[$reference; wikipedia$]$. However, its high cost has limited 
its use in commercial application.

In recent years, many researches use optical sensor as a passive range 
sensor for its low weight, low cost. With the help of computer vision 
technology, optical sensors have been successfully used for range 
mapping and obstacle detection in a number of platforms 
$[$references$]$. There are several types of configuration in using 
optical sensor for range mapping: monocular, binocular, or multi-camera. 
Since optical sensors are bearing only sensors, the principle of range 
measurement is through triangulation a common scene point in two or more 
images captured. For binocular camera setups, two cameras are placed 
apart from each other with their relative geometry known and captures 
images simultaneously. If the position of a scene point can be 
accurately found in the images by both cameras, its distance can be 
calculated by using the difference in position of the projected point in 
images, and the separation of the cameras. 

\begin{itemize}
  \item Radar, sonar, laser range finder, 3D flash lidarvs. optical sensors
  \begin{itemize}
    \item Radar, laser range finder have high power requirement and mass
    \item Depth measurement can be obtained through optical sensors, which 
    are inexpensive and light weight
    \item Depth maps of a 3-D scene can be computed from a single pair of 
    stereo camera. Stereo processing can require significant computational 
    effort
  \end{itemize}
  \item Monocular camera characteristic: 
  \begin{itemize}
    \item bearing-only sensor, which provide the measurement on the 
    direction of the feature, and not the range. Other sensors, such as 
    radar, are range and bearing sensors. 
  \end{itemize}
\end{itemize}

\subsection{Monocular Vision and Binocular Vision}\label{sec:MonoBino}
\begin{itemize}
  \item Optical flow vs. feature detection and tracking
  \item The correspondence problem
  \item Initialization problem (addressed by Inverse depth 
  parameterization)
  \item Lack of scale information of overall map -$>$ must work with other 
  sensors which provide robot motion measurement. 
\end{itemize}

\subsection{Limitation of Optical Sensor in Recursive Algorithm}
\label{sec:OpticalSensorLimitation}

\begin{itemize}
  \item Error Accumulation over Iterations
  \begin{itemize}
    \item Feature quality Decreases over Iterations
  \end{itemize}
\end{itemize}

\subsection{GPS and IMU}\label{sec:gps_and_imu}
GPS and IMU are generally available on UAVs. These sensors provide a 
measurement on the robot motion. Odometry can provide the scale 
information which is missing in the bearing only measurement. 
Furthermore, odometry provides some prior information on the robot 
motion which can help to disambiguate the solution.

\section{SLAM as A Sensor Fusion Framework}
\label{sec:SLAM}
An essential aspect of autonomy for a mobile robot is the capability to 
determine its location. This capability is known as localization. 
Localization is typically a prerequisite for accomplishing real tasks, 
whether it is exploration, navigation toward a known goal, 
transportation of material, construction or site preparation. In many 
applications, the mobile robot has an a priori map. Given a map, the 
robot may localize by matching current sensor observations to features 
in the map. Given enough features and an unambiguous geometry, the pose 
of the robot can be determined or at least narrowed down to a set of 
possible locations.

Usable maps do not always exist, and it is not always possible to have 
accurate externally referenced pose estimates. If an a priori map is not 
available, the robot may need to construct one. With a precise, 
externally referenced position estimate from GPS or similar means, the 
robot can take its sensor observations, reference the observations to 
its current pose, and insert features in the map in the appropriate 
places. Without maps or externally referenced pose information, the 
robot must produce its own map and concurrently localize within that 
map. This problem has been referred to as concurrent localization and 
mapping (CLM) and simultaneous localization and mapping (SLAM).

\subsection{Recursive Probabilistic Estimation using Extended Kalman Filter}
\label{sec:SLAM_using_EKF}

\subsubsection{Kalman Filter}
The Kalman filter \cite{Kalman's original paper}published by R. E.
Kalman in 1960 is a very powerful recursive data processing algorithm
for dynamic stochastic processes. The filter find extensive use in
control and navigation application for its ability of estimating past,
present and even future state. It is an attractive candidate for data
fusion framework as it can process all available measurements
including previous knowledge of the process, regardless of their
precision, to estimate the current value of the variable of interest.
Given a dynamic process that satisfy the assumptions that Kalman
filter is based on, the filter is the optimal algorithm in minimzing
the mean of squared error of the state variable. This section briefly
summerized assumption and formation of Kalman filter that's described
in detail in \cite{from the following}[Sorenson70; Gelb74; Grewal93;
Maybeck79; Lewis86; Brown92; Jacobs93]. A more intuitive introduction
can be found in chapter 1 of [Maybeck79]
%reference in Welch and bishop
% intro to Kalman filter.

The Kalman filter has three assumptions. 1) The system model is
linear. The linearity is deisred in that the system model is more
esily manipulated with engineering tool. When nonlinearities do exist,
the typical approach is to linearize system model at some nominal
points. 2)The noise embedded in system control and measurement is
white. This assumption implies that the noise value is not correlated
in time, and has equal power in all frequency. 3)The probability
density function (PDF) of system and measurement noise is Gaussian. A
Gaussian distribution is fully represented by the first and second
order statistic (mean and variance) of a process. Most other densities
require endless number of orders of statistic to describe the shape
fully. Hence, when the probability density function of a noise process
is non-Gaussian, the Kalman filter that propagates the first and
second order statistic only include some of the information of the
PDF, instead of all, as would be the case with Gaussian noise.

\paragraph{Kalman Filter Models}
The Kalman filter requires two models. The process model define a
discrete-time controlled process by a linear stochastic difference
equation. The $n\times n$ matrix $A$ relates the state variables
$x_{k-1}$ in previous time step $k-1$ to the state variabl $x_{k}$ in
the current time step $k$. The matrix $B$ relates the optional control
input $\mu$ to the state variables $x$. Given measurement vecor $z_k$
of size $1 \times m$, the measurement model relates the state
variables to the measurements by matrix H of size $n \times m$. The
random variable $w$ and $v$ represent the uncertainty or noise of the
process model, and measurement noise. The $w$ and $v$ are assumed to
unrelated to each other, and has Gaussian distribution.

\begin{align}
&\text{Process Model: }x_k = Ax_{k-1}+B\mu_{k-1}+w_{k-1}\\
&\text{Measurement Model: }z_k = Hx_k+v_k
\end{align}



\subsubsection{Properties of SLAM}
\label{sec:SLAM_properties}

Needs editing.Directly from papers.

Dissanayake proved three important convergency properties of the EKF 
solution to SLAM, namely that: (1) the determinant of any submatrix of 
the map covariance matrix decreases monotonically as observations are 
successively made; (2) in the limit as the number of observations 
increases, the landmark estimates become fully correlated and (3) in the 
limit the covariance associated with any single landmark location 
estimate reaches a lower bound determined only by the initial covariance 
in the vehicle location estimate at the time of the first sighting of 
the first landmark.

The properties imply:

\begin{itemize}
  \item The entire structure of the SLAM problem critically depends on 
  maintaining complete knowledge of the cross correlation between landmark 
  estimates. Minimizing or ignoring cross correlations is precisely 
  contrary to the structure of the problem. (Early EKF for OD work 
  eliminate the cross correlations between features and veichel pose in an 
  attempt to reduce computation complexity.)
  \item As the vehicle progresses through the environment the errors in 
  the estimates of any pair of landmarks become more and more correlated, 
  and indeed never become less correlated.
  \item In the limit, the errors in the estimates of any pair of landmarks 
  become fully correlated. This means that given the exact location of any 
  one landmark, the location of any other landmark in the map can also be 
  determined withabsolute certainty.
  \item As the map converges in the above manner, the error in the 
  absolute location of every landmark (and thus the whole map) reaches a 
  lower bound determined only by the error that existed when the first 
  observation was made (Initialize the parameters using 1st frame as 
  coordinate origin with minimum variance - Algorithm initialization).
\end{itemize}
It is important to note that these theoretical results only refer to the 
evolution of the covariance matrices computed by EKF in the ideal linear 
case. They overlook the fact that given that SLAM is a nonlinear 
problem, there is no guarantee that the computed covariance will match 
the actual estimation erros which is the true SLAM consistency issue. 

\subsubsection{Linearization Error and Consistency}
\label{sec:Linearization_error_and_consistency}

Many research report filter divergence due to linearization error. 
Literature review here:
% the citation is a paper that cited the definition from a book.
As defined in \cite{Huang_2008}, a state estimator is consistent
if the estimation errors (i) are zero-mean, and (ii) have
covariance matrix smaller or equal to the one calculated by the filter.

Huang investigate further on properties and consistency of nonlinear 
two-dimentional EKF based SLAM problem, and conclude:

\begin{itemize}
  \item Most of the convergence properties in $[$3$]$ are still true for 
  the nonlinear case provided that the Jacobians used in the EKF equations 
  are evaluated at the true states.
  \item The main reasons for inconsistency in EKF SLAM are due to (i) the 
  violation of some fundamental constraints governing the relationship 
  between various Jacobians when they are evaluated at the current state 
  estimate, and (ii) the use of relative location information from robot 
  to landmarks to update the absolute robot and landmark location 
  estimates.
\end{itemize}

The robot orientation uncertainty plays an important role in both the 
EKF SLAM convergence and the possible inconsistency. In the limit, the 
inconsistency of EKF SLAM may cause the variance of the robot 
orientation estimate to be incorrectly reduced to zero.

Linearization error can be interpreted as error resulted from 
calculating the Jacobian at the estimated state (wrong state) instead of 
the true state. 

\subsubsection{Camera Centric Coordinate System}
\label{sec:cam_centric}

\subsubsection{SLAM for Large Scale Maps}

%%% Local Variables:
%%% mode: latex
%%% TeX-master: "thesis"
%%% End:

