\chapter{Introduction} \label{ch:intro}
% Word checked, no spell and gramma error. 
For any autonomous vehicle, generating an accurate and high precision
model of its surrounding environment to indicate hazard features, and
knowing its own location in the map is essential for the vehicle to
navigate and avoid obstacle autonomously.

In many applications, the mobile robot has a priori map. The given
priori map may be sufficient for localization purpose, but generally
do not have sufficient resolution or up-to-date information for
obstacle detection. Ground vehicles need to deal with temporary added
road block and parked cars. Aerial vehicles need high resolution map
that indicates tall trees, steep hills or electrical towers. In
addition, usable map does not always exist. Without maps and
externally referenced pose information, the robot must produce its own
map and concurrently localize itself within the map. This problem is
referred to as the simultaneous localization and mapping (SLAM).

Traditional two-dimensional SLAM algorithms are well established in
the past decade. A SLAM algorithm typically utilises measurements from
several types of sensor which can be divided into two groups, those
that provide vehicle poses measurements, such as wheel odometer, or
GPS; and those that provide landmarks bearing and range measurements,
such as radar, sonar, laser range finder. In recent years, optical
sensors are actively being incorporated into SLAM algorithm and
successfully used in ground vehicle navigation. For aerial vehicles,
the experiments are mostly limited to
simulation\cite{nemra_robust_2010} \cite{jianli_unscented_2011}
\cite{sunderhauf_using_2007} \cite{artieda_visual_2009}, and results
from realistic aerial video data are rare.

\section{Problem Statement}\label{section:ProblemStatement}
% Word checked.
Obstacle detection (OD) has received a lot of research interest in
recent years. Various algorithms were developed for ground, under
water and aerial vehicle using different sensors such as sonar, radar,
LIDAR, and vision. Most OD system focused on only one sensor. Yet,
using multiple sensors generally produced better measurements than
single sensor \cite{smith_approaches_2006}. On most unmanned aerial
vehicle (UAV) platforms, many sensors are readily available ,such as
accelerometers, gyroscope, GPS receiver, altimeter, etc. Fully
utilizing these sensors should improve the accuracy and robustness of
an OD system, especially in harsh flying condition.

This thesis focused on developing an obstacle detection system by
using SLAM algorithm as sensor fusion framework to integrate
measurements from various sensors on a typical UAV navigation device.
The type of application targeted by this work is medium size UAV
conducting low altitude terrain following flight in natural
environment. The obstacles are static objects on ground; moving
objects are not considered. Research presented in this thesis
contributed to the project of developing a mid-size UAV to perform
geological survey, carried out by Carleton University in collaborated
with Sander Geophysics Ltd. who is an Ottawa based company specialized in
high precision geophysical survey. To achieve high resolution data
acquisition, the UAV must be able to perform terrain following flight
with altitude as low as 10 meters from ground, and with ground speed
ranging from 60 knots (30 m/s) to 100 knots (51.4 m/s). The specified
rate of climb for the UAV is 400ft of vertical rise per minutes (122
meters per minutes) \cite{james_geosurv_2008}. A quick analysis on the
UAV specification and aerodynamic behavior reveals the detection
requirement of the OD system. Assuming a tree height of 20 meters,
which is the average height for oak or pine, to allow for enough time
to avoid obstacle, the UAV must be able to detect the threat at 610
meters or further (Figure \ref{ob}). This analysis
indicated that the obstacle detection must be able to map objects up to
a thousand meters from the UAV.

\begin{figure}[h]
\centering
\includegraphics[width=300pt,height=160pt]{./Figures/ProblemStatement.png}
\caption {Case study for obstacle detection requirement}
\label{ob}
\end{figure}

Although digital terrain map are generally available for flight path
planning and in flight navigation, it does not have the resolution to
indicate all hazardous obstacles such as tall trees, steep hills, or
man-made objects. The obstacle detection and avoidance system must be
in place to detect discrete threat, and to operate automatically with
minimum intervene from operator.

\section{Contributions}\label{section:Contribution}
% word checked.
The thesis first reviewed pros and cons of various sensors, and
pointed out the advantage and disadvantage of using imaging sensor in
OD application. Different types of imaging sensor configurations and
sensor calibration methods were also described. Next, the thesis
reviewed the formulation and properties of a typical Extended Kalman
Filter (EKF) based SLAM algorithm, and discussed the advantage and
limitation of a EKF-based SLAM algorithm.

The reviews and discussions led to implementation of an improved EKF
SLAM method by fusing multiple sensors with monocular camera video.
%Using a monocular vision for mapping is a bearing only
%problem. The measurement is through projection, which loses
%information about the relative position of the feature since the range
%is unknown. Without camera motion measurements, map created by
%monocular vision can be scaled arbitrarily. 
A camera centric EKF based SLAM algorithm (referred to as CC-EKF-SLAM
in the rest of the article) was described in this thesis. The
algorithm utilized an extended Kalman filter to fuse camera motion
measurements from inertial and gyroscope sensors and landmarks
measurements from video of a single wide angle camera. Inverse
parameterization was adopted to describe the landmarks positions so
that distanced landmarks can be estimated. Camera centric coordinate
was used to improve the consistency of the framework for large area
mapping. The filter can estimate absolute coordinates of landmarks and
position of the UAV directly. An interpolated map can be generated
from the estimates to represent the surrounding environment of the
UAV, and the location of the UAV within.

To test the algorithm under true flying condition, aerial flight data
were collected and processed by the CC-EKF-SLAM algorithm. To capture
low altitude flight video and navigation data, a simulated unmanned
aerial system (SUAS) with various sensors onboard was towed by a
helicopter which flew a pre-planned path in the mountain north of
Gatineau, Quebec. Sensors installed included 1 CCD camera with 6mm
focal length lens, GPS antenna, a UAV navigation module with embedded
accelerometer, gyroscope and external fluxgate sensor. The helicopter
flew at speed of 60 knots, at altitude of 100 meters above ground,
with the SUAS at approximately 70 meters above ground. Two pieces of
video and data were processed by the CC-EKF-SLAM algorithm. The result
proved that CC-EKF-SLAM algorithm was capable of mapping landmarks
over 1000 meters away. The preliminary result of the test flight was
published in \cite{zhang_obstacle_2012}. This paper was one of the
first in the field that successfully applying monocular vision SLAM in
large scale aerial OD application.

To further analyze error seen in the flight test result, a series
of simulations were done to thoroughly study the behavior of the
algorithm under various circumstances, including
\begin{itemize}
\item UAV conducting simple forward motion
\item UAV experiencing oscillatory motion on all other 5 degree of
freedom (DOF)
\item error existed in camera calibration result
\item quantization error introduced through image digitization 
\end{itemize}
The results of these simulations were valuable to the design of data
acquisition hardware for obstacle detection purpose, and to the future
improvement of the CC-EKF-SLAM algorithm.

\section{Organization}\label{section:Organization}
% word checked
The thesis was organized as follows:

\begin{itemize}
  \item Chapter 2 presented an overview on sensors, and data fusion
  framework (EKF and SLAM) related to obstacle detection and range
  measurement.
  \item Chapter 3 described experiment setup for the aerial data
  collection, camera calibration procedure and result, and ground
  truth data collection.
  \item Chapter 4 described the detail implementation of the proposed
  CC-EKF-SLAM algorithm. Data preparation steps used to compare
  estimated data with the ground truth was also given in this chpater.
  \item Chapter 5 presented the result of flight test. Convergence and
  consistency of the algorithm were discussed. Accuracy analysis by
  comparing to the ground truth data was also presented.
  \item Chapter 6 presented the result of error analysis through
  simulations. Behavior of the CC-EKF-SLAM
  algorithm was studied through simulating a number of scenario.
  \item Chapter 7 gave an overall summary of the result obtained in
  this research, and recommendations for future research.
\end{itemize}

%%% Local Variables:
%%% mode: latex
%%% TeX-master: "thesis.tex"
%%% End:
