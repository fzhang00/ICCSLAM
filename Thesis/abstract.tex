\doublespacing
This thesis describes the research of an obstacle detection system for a low flying
autonomous unmanned aerial vehicle (UAV). The system utilized an
extended Kalman filter based simultaneous localization and mapping
algorithm which fuses navigation measurements with
monocular image sequence to estimate the poses of the UAV and the positions
of landmarks.

To test the algorithm with real aerial data, a test flight was conducted to
collect data by using a sensors loaded simulated unmanned aerial system(SUAS) towed
by a helicopter. 
The results showed that the algorithm is capable of mapping
landmarks ranging more than 1000 meters. Accuracy analysis also showed that SUAS localization and landmark mapping
results generally agreed with the ground truth. 

To better understand the strength and weakness of
the system, and to improve future designs, the
algorithm was further analyzed through a series of simulations which simulates oscillatory motion of the UAV, error embedded in camera calibration result, and quantization error from image digitization.
