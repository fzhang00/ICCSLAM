% first sentence describes what this algorithm is for, and what it is
% made of.
This thesis describes an obstacle detection system for low flying
autonomous unmanned aerial vehicle (UAV) using a simultaneous
localization and mapping (SLAM) algorithm.
% describe the algorithm and it's output 
The SLAM framework utilizes an extended Kalman filter to fuse inertial
measurements and monocular image sequence together to estimate the
poses of the UAV and positions of landmarks. A high resolution sparse
terrain map and UAV trajectory can be computed from the filter state
vector. 
\emph{(optional description of the algorithm)In order to
detect long distance obstacle and handle large area mapping, inverse
depth parameterization and camera centric coordinate system was
adopted.}
% Test flight and result
The biggest contribution of this work is that the algorithm was tested
and proven feasible with real arial video and navigation measurements.
A test flight was conducted to collect data by using a simulated
unmanned arial system (SUAS) towed by a helicopter. The result shows
that when quality features can be extracted from the image sequence,
the algorithm is capable of mapping landmarks ranging as far as 1600
meters. The accuracy of the landmark position estimates were good for
those initialized at the first frames. Landmarks added later carried
offset error due to the error in aircraft localization estimates.
% The simulations
To better understand the errors seen in the real flight data result, a
series of simulations were performed to analyze the algorithm. The
simulated scenarios include simple forward motion, oscillatory motion
on all other 5 degrees of freedom, the effect of camera calibration
error, and the effect of various image resolution. The simulation
showed that the algorithm is very sensitive to rotational motion,
which caused significant error in both landmark position estimates and
UAV localization estimates. Other improvements suggested by the
simulation include adding len distortion model into the algorithm, and
using camera with resolution higher than 720x1080.
