% first sentence describes what this algorithm is for, and what it is
% made of.
This thesis describes an obstacle detection system for low flying
autonomous unmanned aerial vehicle (UAV) using a simultaneous
localization and mapping (SLAM) algorithm.
% describe the algorithm and it's output 
The SLAM framework utilizes an extended Kalman filter to fuse inertial
measurements and monocular image sequence together to estimate the
poses of the UAV and positions of landmarks. A high resolution sparse
terrain map and UAV trajectory can be computed from the filter state
vector. 
\emph{(optional description of the algorithm)In order to
detect long distance obstacle and handle large area mapping, inverse
depth parameterization and camera centric coordinate system was
adopted.}
% Test flight and result 
The biggest contribution of this work is that the algorithm was tested
and was proven feasible with real aerial video and navigation measurements.
A test flight was conducted to collect data by using a modeled UAV towed
by a helicopter. The result showed that when good quality features can
be extracted from the image sequence, the algorithm was capable of
mapping landmarks ranging as far as 1600 meters when they were
initialized at the first frame of the video. Landmarks added later
carried offset errors due to the error in aircraft localization
estimates.
% The post analyzation 

To better understand its strength and weakness, and to improve future
design, the algorithm was further analyzed through a series of
simulation. The simulated scenarios were designed to
reconstruct what was seen in the real flight and included: simple
forward motion, oscillatory motion on all other 5 degrees of freedom,
the effect of camera calibration error, and the effect of image
digitization. The simulation showed that the algorithm is very
sensitive to rotational motion, which caused significant error in both
landmark position estimates and UAV localization estimates. Other
improvements suggested by the simulation include adding lens
distortion model into the algorithm, and using camera with resolution
higher than 720x1080.
