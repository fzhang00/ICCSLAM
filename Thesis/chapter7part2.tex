% chapter 6 summary
Chapter 6 describes a series of simulation done in the goal of
better understanding the noise source in CC\_EKF\_SLAM algorithm. An
nearly no noise environment was first simulated with SUAS moving
forward only, with no error introduced from image digitization or
camera model mismatch. Effect from various SUAS motion was simulated
next. Then various image resolution settings, and camera model mismatch were
simulated. 

Under forward only motion, SUAS translation error is less than 1cm,
orientation error under $3e-3^\circ$; feature position error is under
0.2 meters for X component, and under 0.02 meters for Y and Z
components. The result above shows error introduced by the algorithm
itself under simple forward motion is very minor.

The effect from complex SUAS motion was analyzed next. The SUAS remain
forward traveling. The other 5 d.o.f. motion was simulated with a 1Hz
sine wave with variable amplitude setting, and added on top of the
forward motion one at a time. Result shows that translational motion
and rotation around X axis has little effect on the SUAS localization
accuracy. Rotation around Y axis (pitch) and rotation around Z axis
(azimuth) causes SUAS positioning error on Z and Y respectfully to up to
30 meters peak-to-peak depending on the magnitude of the rotation. For
feature mapping accuracy, translation motion
