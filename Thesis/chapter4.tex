\chapter{Experiments with Real Aerial Data}
\section{Equipment Setup and Flight Data Collection}
% word checked.
The biggest contribution of the project is that the proposed algorithm
was tested and proven feasible with real flight data collected by
author. The aerial video and navigation data were collected through a
survey flight with the support of Sander Geophysics Ltd. A main
purpose of the test flight is to obtain aerial video with the camera
close to the ground to mimic the scenario of a low flying UAV. This is
difficult to achieve with any manned fixed wing aircraft since terrain
following flight at low altitude is very dangerous. Therefore, a
helicopter was used to conduct the survey flight to achieve the
terrain following action. To get sensors close to the ground and to
allow for flexible sensors mounting, a simulated unmanned aircraft
system (SUAS) was used to carry all sensors. The SUAS was towed by a
helicopter via a tow rope of 33 meters long (Figure
\ref{fig:towedSUAS}). Yet, sufficient clearance must be established
between the SUAS and the vegetation to prevent the SUAS from being
caught by tree branches. As a result, the helicopter flew a planned
path at approximately 100 meters above ground, and the SUAS was at
approximately 70 meters above ground.

\begin{figure}[h]
\centering
\includegraphics[width=6cm,keepaspectratio=true]{./Figures/SUAS_TAKEOFF2.jpg}
\includegraphics[width=6cm,keepaspectratio=true]{./Figures/SUAS_TAKEOFF3.jpg}
\includegraphics[width=6cm,keepaspectratio=true]{./Figures/SUAS_TAKEOFF4.jpg}
\includegraphics[width=6cm,keepaspectratio=true]{./Figures/towed_SUAS.jpg}
\caption{Simulated UAS towed by helicopter}
\label{fig:towedSUAS}
\end{figure}
\FloatBarrier

Sensors mounted on the SUAS included one wide angle CCD camera with
6mm focal length lens capturing monocular image sequence at 30 fps, a
pair of narrow angle CCD cameras for binocular images, one GPS
antenna, and one flight control INS/GPS navigation unit Athena GS-111m
\cite{_athena_????} (Figure \ref{fig:SUAS}) which captures vehicle
velocity, acceleration, rotation, sea level altitude, etc. Analog
video and navigation data were sent to the helicopter via three BNC
cables and one data cable for recording. Installed in the helicopter
were two SGL data acquisition systems CDAC. This system records video
and navigation data from the SUAS, as well as data from sensors
installed on the helicopter, including GPS, radar altimeter, laser
altimeter, air pressure, temperature, humidity, etc. (figure
\ref{fig:CDAC}). Videos from the three cameras were digitized to a
resolution of 720 pixels by 480 pixels using an Parvus MPEG4 video
encoder installed in CDAC. Video were time-stamped with GPS second on
the image screen \ref{fig:video_snapshot} for post-flight
synchronization with the navigation data. Because the test flight
recorded data from all available sensors on board the SUAS and the
helicopter, should any future research require more sensor data, these
recordings would still be useful.

\begin{figure}[h]
  \centering
  \includegraphics[width=14cm,keepaspectratio=true]{./Figures/SUAS.jpg}
  \caption{Sensors mounting location on SUAS}
  \label{fig:SUAS}
\end{figure}

\begin{figure}[h]
  \centering
  \includegraphics[width=6cm,keepaspectratio=true]{./Figures/athena.jpg}
  \includegraphics[width=6cm,keepaspectratio=true]{./Figures/GPS_antenna.jpg}
  \includegraphics[width=10cm,keepaspectratio=true]{./Figures/wide_cam.jpg}
  \caption{Sensors mounted on SUAS. Top left: Athena
  GS-111m, top right: GPS antenna, bottom: monocular CCD camera}
  \label{fig:SUAS_sensors}
\end{figure}

\begin{figure}[h]
  \centering
  \includegraphics[width=12cm,keepaspectratio=true]{./Figures/CDAC_Rack.jpg}
  \caption{Compact PCI data acquisition system (CDAC)}
  \label{fig:CDAC}
\end{figure}

\begin{figure}[h]
  \centering
  \includegraphics[width=12cm,keepaspectratio=true]{./Figures/video_snapshot.jpg}
  \caption{Image from monocular camera with GPS second timestamp}
  \label{fig:video_snapshot}
\end{figure}

\FloatBarrier

\section{Camera Calibration}\label{sec:camcal}
% word checked
Camera calibration decodes the relation between image pixels and the
actual 3D world. The intrinsic parameters could be affected by a
number of environmental conditions, such as temperature, and humidity.
To extract the camera parameters at a camera condition as close as
possible to the one during test flight, camera calibration was
performed right after the SUAS returned to the SGL hanger. A camera
calibration was done by taking a video of a checker board pattern with
various translation and rotation from the camera. A total of 20 views
of the calibration target were chosen from the video, and fed to the
calibration algorithm. A few examples are shown in figure
\ref{fig:camcal}.

\begin{figure}[h]
  \centering
  \includegraphics[width=4cm,keepaspectratio=true]{./Figures/camcal/camcal1_130.jpeg}
  \includegraphics[width=4cm,keepaspectratio=true]{./Figures/camcal/camcal1_140.jpeg}
  \includegraphics[width=4cm,keepaspectratio=true]{./Figures/camcal/camcal1_160.jpeg}
  \includegraphics[width=4cm,keepaspectratio=true]{./Figures/camcal/camcal1_180.jpeg}
  \includegraphics[width=4cm,keepaspectratio=true]{./Figures/camcal/camcal1_210.jpeg}
  \includegraphics[width=4cm,keepaspectratio=true]{./Figures/camcal/camcal1_240.jpeg}
  \caption{A subset of camera calibration input images}
  \label{fig:camcal}
\end{figure}

The program ``calibration.exe'' was used to calibrate the camera. This program came with the OpenCV installation. The digitized images have a resolution of 720 pixels in width and 480 pixels in height. Table \ref{tab:camcalresult} below lists the calibration results.

\begin{table}[h]
\caption{Camera calibration result}
\label{tab:camcalresult}
\centering
\begin{tabular}{|c|c|}
\hline
Parameter & Result\\ \hline
$f_x$ & 887.6 pixels \\ \hline
$f_y$ & 805.7 pixels\\ \hline
$c_x$ & 381.8 pixels\\ \hline
$c_y$ & 293.7 pixels\\ \hline
$k_1$ & -0.102 \\ \hline
$k_2$ & -0.535 \\ \hline
$p_1$ & 1.15e-003 \\ \hline
$p_2$ & 8.40e-003 \\
\hline
\end{tabular}
\end{table}
\FloatBarrier

\section{Ground Truth Data Collection}
The localization ground truth was obtained through the flight
control unit GS-111m on board the SUAS. The unit recorded the SUAS
position in GPS longitude and latitude coordinate. Orientation was
obtained from the roll pitch and heading measurements. Roll and pitch
accuracy has $0.1^\circ$ and $0.1^\circ$ in standard deviation.
Heading accuracy can achieve $0.5^\circ$ \cite{_athena_????}.

Landmark position ground truth came from digital elevation map (DEM)
downloaded from CGIAR-CSI website \cite{_cgiar-csi_????}. The DEM
contains longitude, latitude and sea level elevation of the terrain
with a resolution approximately 100 meters by 100 meters. 

%consider adding the ground truth DEM figure here. 


%%% Local Variables:
%%% mode: latex
%%% TeX-master: "thesis.tex"
%%% End:
