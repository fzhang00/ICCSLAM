\message{ !name(thesis.tex)}\documentclass[12pt,letterpaper]{report}
%
% Review cuthesis.sty for more documentation on available options
% for this package.

% Leave the below option as "masters," as it generates "thesis" on title page. 
% PhD generates "dissertation" and that word is not recognized by Carleton Library
\usepackage[masters,2committee]{cuthesis}
\usepackage{array}
\usepackage{graphicx}
\usepackage{cite}
\usepackage{subfigure}
\usepackage{amsmath}
\usepackage[indent,bf]{caption}
\usepackage{rotating}
\usepackage{setspace}
\usepackage{longtable}
\usepackage{rotating}




% Defines relative path to folder containing your figures
\graphicspath{{figures/}}
\providecommand{\norm}[1]{\lvert#1\rvert} 

\begin{document}

\message{ !name(chapter6.tex) !offset(-15) }

    \end{itemize}
    \item Image resolution.
    \item Accelerometer bias 
  \end{itemize}
  \item Error introduced by LK tracking algorithm. Reliable vision
  tracking is an entire field of research in itself. Pyramid
  implementation of Lucas-Kanade (LK) tracking is used in this work,
  and there are a number of factors contribute to its performance.
  Firstly, LK tracking algorithm tracks features by comparing the
  intensity of a window of the image centered at the feature
  coordinate in image from one frame to another. The searching process
  terminates when the sum of error on the windowed image intensity is
  lower than a value set by user, or when iteration of search has
  reached a maximum number set by user. Secondly, as scene evolve from
  frame to frame, the initial feature appears differently from frame
  to frame as the viewing distance and angle is different. Thirdly,
  sudden intensity change in the image sequences significant noise in
  the tracking. In outdoor setting, intensity change can be introduced
  by many factors, such as changes of sky area in a image, sun glare,
  UAV enters or exits cloud shades, or camera auto-adjust its shuttle
  speed, etc.
  \item Error caused by the SLAM algorithm itself. The algorithm
  estimated features coordinate through a model that represents the
  relation between UAS location, feature location and UAS motion. As
  the model is non-linear, the linearization process introduces error
  into the result.
\end{enumerate}

To better understand the impact of the factors listed above. A 
simulation is performed to examine the impact item 1 and item 3. 

The simulator first generates a 3D point cloud ranging from 100 meters
to 3000 meters from the camera (Figure \ref{fig:simfig51}). At each
frame, the coordinates of the 3D points are first transformed to the
new camera frame using the measured UAS motion. Next, the 3D points
are projected to the image plane using a camera model defined by

\message{ !name(thesis.tex) !offset(33) }

\end{document}

%%% Local Variables: 
%%% mode: latex
%%% TeX-master: t
%%% End: 
