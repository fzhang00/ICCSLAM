\chapter{Jacobian Matrix for Filter Initialization Equations}\label{ch:appendix2}
The Jacobian matrix used in the state covariance matrix initialization
Equation (\ref{eq:Pnew}) is derived in this section. The complete
Jacobian matrix $\boldsymbol{J}$ for initializing a landmark
covariance is given by

\begin{equation}
\boldsymbol{J}=\begin{bmatrix}
 & & & & & &\boldsymbol{0}\\
 & &\boldsymbol{I}& & & &\vdots\\ 
 & & & & & &\boldsymbol{0}\\
\boldsymbol{\frac{\partial p_{i}}{\partial OX_{W}^{C}}} &
\boldsymbol{\frac{\partial p_{i}}{\partial c^{C}}} & 
\boldsymbol{\frac{\partial p_{i}}{\partial r^{C}}} & 
\boldsymbol{0} & \ldots & \boldsymbol{0} & 
\boldsymbol{\frac{\partial p_{i}}{\partial g_{i}}}
\end{bmatrix}
\end{equation}

\noindent Whenever a new landmark is added, its initial position is
always $[\begin{matrix}0&0&0\end{matrix}]$ in the camera frame, and
the $[\begin{matrix}\rho&\varphi&\theta\end{matrix}]$ parameters are
not a function of $ \boldsymbol{OX_{W}^{C}}$, $\boldsymbol{c^{C}}$,or
$\boldsymbol{r^{C}}$, therefore

\begin{equation}
\begin{matrix}
\boldsymbol{\frac{\partial p_{i}}{\partial OX_{W}^{C}}}=\boldsymbol{0_{6\times 6}}, & 
\boldsymbol{\frac{\partial p_{i}}{\partial c^{C}}}=\boldsymbol{0_{6\times 3}}, &
\boldsymbol{\frac{\partial p_{i}}{\partial r^{C}}}= \boldsymbol{0_{6\times 3}}
\end{matrix}
\end{equation}

\noindent $\boldsymbol{J}$ can then be simplified as

\begin{equation}
\boldsymbol{J}=\begin{bmatrix}
\boldsymbol{I} & \boldsymbol{0} & \\
\boldsymbol{0} & \boldsymbol{\frac{\partial p_{i}}{\partial g_{i}}} & \\
\end{bmatrix} 
\end{equation}


\noindent where $\boldsymbol{g_{i}}$ includes the variable in matrix $\boldsymbol{R}$ given in
Equation(\ref{eq:R}): $\boldsymbol{g_{i}}=[\begin{matrix} x_{i}^{C} & y_{i}^{C} &
  z_{i}^{C} & \rho _{i} & u_{i} & v_{i}\end{matrix}]$. Then 

\begin{equation}
\label{eq:b4}
\boldsymbol{\frac{\partial p_{i}}{\partial g_{i}}}=\begin{bmatrix}
\boldsymbol{I_{3\times 3}} & & \boldsymbol{0_{3\times 3}} & \\
 & \frac{\partial \rho_{i}}{\partial \rho_{i}} & 
\frac{\partial \rho_{i}}{\partial u_{i}} & 
\frac{\partial \rho _{i}}{\partial v_{i}}\\
\boldsymbol{0_{3\times 3}} & \frac{\partial \varphi_{i}^{C}}{\partial \rho_{i}} & 
\frac{\partial \varphi_{i}^{C}}{\partial u_{i}} & 
\frac{\partial \varphi_{i}^{C}}{\partial v_{i}} \\
 & \frac{\partial \theta_{i}^{C}}{\partial \rho_{i}} & 
\frac{\partial \theta_{i}^{C}}{\partial u_{i}} & 
\frac{\partial \theta_{i}^{C}}{\partial v_{i}} \\
\end{bmatrix} = \begin{bmatrix}
\boldsymbol{I_{3\times 3}} & & \boldsymbol{0_{3\times 3}} & \\
 & 1 & 0 & 0 \\
\boldsymbol{0_{3\times 3}} & 0 & \frac{\partial \varphi _{i}^{C}}{\partial u_{i}} 
& \frac{\partial \varphi_{i}^{C}}{\partial v_{i}} \\
 & 0 & \frac{\partial \theta_{i}^{C}}{\partial u_{i}} & 
\frac{\partial \theta_{i}^{C}}{\partial v_{i}} \\
\end{bmatrix}
\end{equation}

\noindent Based on the chain rule of differentiation, the partial
derivatives in Equation \ref{eq:b4} is given by

\begin{equation}
\label{eq:b5}
\frac{\partial \varphi_{i}^{C}}{\partial u_{i}}=
\boldsymbol{\frac{\partial \varphi_{i}^{C}}{\partial h^{C}}\frac{\partial h^{C}}{\partial u_{i}}}
\end{equation}

\begin{equation}
\label{eq:b6}
\frac{\partial \theta _{i}^{C}}{\partial u_{i}}=
\boldsymbol{\frac{\partial \theta_{i}^{C}}{\partial h^{C}}\frac{\partial h^{C}}{\partial u_{i}}}
\end{equation}

\begin{equation}
\label{eq:b7}
\frac{\partial \varphi _{i}^{C}}{\partial v_{i}}=
\boldsymbol{\frac{\partial \varphi _{i}^{C}}{\partial h^{C}}\frac{\partial h^{C}}{\partial v_{i}}}
\end{equation}

\begin{equation}
\label{eq:b8}
\frac{\partial \theta _{i}^{C}}{\partial v_{i}}=
\boldsymbol{\frac{\partial \theta_{i}^{C}}{\partial h^{C}}\frac{\partial h^{C}}{\partial v_{i}}}
\end{equation}

\noindent When lens distortion is ignored,
Equation \ref{eq:b5}-\ref{eq:b8} can be calculated using Equation
\ref{eq:par_varphi_h} through \ref{eq:par_h_v}.

\begin{equation}
\label{eq:par_varphi_h}
\boldsymbol{\frac{\partial \varphi_{i}^{C}}{\partial h^{C}}}= 
\begin{bmatrix}
\frac{\partial \varphi_{i}^{C}}{\partial h_{x}^{C}} & 
\frac{\partial \varphi_{i}^{C}}{\partial h_{y}^{C}} & 
\frac{\partial \varphi_{i}^{C}}{\partial h_{z}^{C}}
\end{bmatrix}=\begin{bmatrix}
\frac{-h_{x}^{C}\cdot h_{z}^{C}}{({h_{x}^{C}}^{2}+{h_{y}^{C}}^{2}+{h_{z}^{C}}^{2})\sqrt{{h_{x}^{C}}^{2}+{h_{y}^{C}}^{2}}} \\
\frac{-h_{y}^{W}\cdot h_{z}^{W}}{({h_{x}^{C}}^{2}+{h_{y}^{C}}^{2}+{h_{z}^{C}}^{2})\sqrt{{h_{x}^{C}}^{2}+{h_{y}^{C}}^{2}}} \\
\frac{\sqrt{{h_{x}^{C}}^{2}+{h_{y}^{C}}^{2}}}{({h_{x}^{C}}^{2}+{h_{y}^{C}}^{2}+{h_{z}^{C}}^{2})}
\end{bmatrix}^{T}
\end{equation}

\begin{equation}
\label{eq:par_theta_h}
\boldsymbol{\frac{\partial \theta_{i}^{C}}{\partial h^{C}}}=\begin{bmatrix}
\frac{\partial \theta_{i}^{C}}{\partial h_{x}^{C}} & 
\frac{\partial \theta_{i}^{C}}{\partial h_{y}^{C}} & 
\frac{\partial \theta_{i}^{C}}{\partial h_{z}^{C}} \end{bmatrix}
=\begin{bmatrix}
-\frac{h_{y}^{C}}{{h_{x}^{C}}^{2}+{h_{y}^{C}}^{2}} & 
\frac{h_{x}^{C}}{{h_{x}^{C}}^{2}+{h_{y}^{C}}^{2}} & 0
\end{bmatrix}
\end{equation}

\begin{equation}
\boldsymbol{\frac{\partial h^{C}}{\partial u_{i}}}=\begin{bmatrix}
\frac{\partial h_{x}^{C}}{\partial u_{i}} \\
\frac{\partial h_{y}^{C}}{\partial u_{i}} \\
\frac{\partial h_{z}^{C}}{\partial u_{i}} \\
\end{bmatrix}= \begin{bmatrix}
0 \\
s_{x}\\
0
\end{bmatrix}
\end{equation}

\begin{equation}
\label{eq:par_h_v}
\boldsymbol{\frac{\partial h^{C}}{\partial v_{i}}}=\begin{bmatrix}
\frac{\partial h_{x}^{C}}{\partial v_{i}}\\
\frac{\partial h_{y}^{C}}{\partial v_{i}}\\
\frac{\partial h_{z}^{C}}{\partial v_{i}}\\
\end{bmatrix}= \begin{bmatrix}
0 \\
0 \\
s_{y}\\
\end{bmatrix}
\end{equation}

\noindent Equations given above completes the Jacobian matrix calculation.

%%% Local Variables:
%%% mode: latex
%%% TeX-master: "thesis"
%%% End:
